
\documentclass[11pt]{article}
\usepackage{ndstyle}
\setcounter{secnumdepth}{2}
\setcounter{tocdepth}{2}

\title{New Doggerland \\ Governance Schedules \\ Schedule D - Logging and Audit Integrity Policy}
\author{}
\date{}

\begin{document}
\maketitle
\tableofcontents
\newpage

\noindent
\textbf{Status:} Binding Governance Policy \\
\textbf{Authority:} Adopted by the Governing Body pursuant to the Consolidated Governance Document \\
\textbf{Purpose:} Define the logging, auditability, and tamper-evident requirements for all governed actions \\
\textbf{Precedence:} Subordinate to the Consolidated Governance Document; controlling where referenced \\
\textbf{Mutability:} Revisable only by Extraordinary Process

\section{Purpose}

This Policy defines the minimum technical and procedural requirements for logging, record integrity, auditability, and retention across all systems governed by the New Doggerland governance framework.

No action requiring logging under the Consolidated Governance Document may occur unless it can be logged in compliance with this Policy.

\section{Scope}

This Policy applies to:
\begin{itemize}
\item all Recorded Decision Pathways,
\item all Material Actions,
\item all emergency actions,
\item all governance votes and approvals,
\item any action explicitly required to be logged by policy or schedule.
\end{itemize}

\section{Core Logging Requirements}

All governed log entries must include, at minimum:

\begin{itemize}
\item a unique entry identifier,
\item timestamp (UTC),
\item actor identity and role,
\item authority relied upon,
\item action description,
\item affected scope objects (if any),
\item linkage to related entries.
\end{itemize}

\section{Primary Logging System}

The Primary Logging System must satisfy all of the following:

\begin{itemize}
\item append-only write model,
\item immutable historical entries,
\item cryptographic or checksum-based integrity verification,
\item role-based access control,
\item read access for authorized reviewers,
\item export capability for audit.
\end{itemize}

No user may delete or overwrite a primary log entry.

\section{Edit and Correction Protocol}

Corrections to logged information must be made only by:

\begin{itemize}
\item creating a new log entry,
\item referencing the original entry identifier,
\item stating the correction and reason.
\end{itemize}

The original entry must remain intact and visible.

\section{Offline and Emergency Logging}

If the Primary Logging System is unavailable, actions may proceed only where explicitly permitted by policy.

In such cases:

\begin{itemize}
\item an offline log stub must be created immediately,
\item the stub must be physically or digitally tamper-evident,
\item the stub must be ingested into the Primary Logging System within 72 hours of restored capability.
\end{itemize}

Where an Operational Manual specifies a shorter or longer ingestion window, the window defined in this Schedule controls.

Failure to ingest an offline stub constitutes a violation.

\section{Tamper-Evident Standards}

A logging mechanism is considered tamper-evident only if:

\begin{itemize}
\item alterations are detectable,
\item access is restricted and logged,
\item integrity verification can be independently validated.
\end{itemize}

Procedures or systems that rely on trust alone do not satisfy this requirement.

\section{Access Controls and Segregation of Duties}

No individual may:

\begin{itemize}
\item both execute and approve the same logged action,
\item both administer the logging system and authorize actions logged therein.
\end{itemize}

Access rights must be reviewed at least annually and upon role change.

\section{Retention and Backup}

All logs must be retained for a minimum of seven years unless a longer period is required by law or policy.

Backup requirements:

\begin{itemize}
\item regular automated backups,
\item off-site or logically isolated storage,
\item periodic restoration testing.
\end{itemize}

\section{Audit and Review}

Authorized auditors may:

\begin{itemize}
\item access read-only log exports,
\item verify integrity checks,
\item trace Recorded Decision Pathways end-to-end.
\end{itemize}

Obstruction of audit access constitutes a governance violation.

\section{Interpretive Rule}

This Policy must be interpreted narrowly. Where ambiguity exists, interpretation must favor:

\begin{itemize}
\item greater auditability,
\item non-reversibility of log history,
\item escalation rather than omission.
\end{itemize}

\section{Amendment}

This Policy may be amended only through an Extraordinary Process as defined in the Consolidated Governance Document.

All amendments must be recorded, versioned, and published alongside prior versions.

\end{document}
