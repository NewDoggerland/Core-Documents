\documentclass[11pt]{article}
\usepackage{ndstyle}
\setcounter{secnumdepth}{3}
\setcounter{tocdepth}{3}

\title{New Doggerland - Board Vetting Process}
\author{}
\date{}

\begin{document}
\maketitle

\section{Status and Authority}

\noindent
\textbf{Status:} Non-binding evaluative procedure \\
\textbf{Authority:} Subordinate to \emph{New Doggerland -- Consolidated Governance Document v0.5.2} \\
\textbf{Purpose:} Assess eligibility and screen disqualifications for prospective Governing Body members in a manner consistent with the Constitution.

This document derives authority solely from the Consolidated Governance Document. It does not grant authority, create obligations, modify eligibility thresholds, or override any constitutional provision.

Failure to follow this process does not validate an ineligible appointment and may constitute a governance process failure.

\section{Scope and Limitations}

This vetting process:
\begin{itemize}
\item evaluates eligibility under the Constitution,
\item documents evidence relied upon,
\item does not substitute for removal, classification, or enforcement procedures,
\item does not limit Governing Body discretion within constitutional bounds.
\end{itemize}

This process must not be treated as an independent source of authority.

\section{Vetting Stages}

Vetting must proceed in the order below. Failure at any stage terminates the process.

\subsection{Stage 1 -- Independence and Disclosure Screening}

The candidate must provide written disclosure confirming:
\begin{itemize}
\item ability to act independently,
\item absence of prohibited relationships,
\item disclosure of all Material Financial Interests,
\item acknowledgment of a continuing disclosure obligation.
\end{itemize}

\noindent
\texttt{IF any disqualifying relationship or undisclosed conflict exists, THEN terminate vetting.} \\
\texttt{ELSE proceed to Stage 2.}

\subsection{Stage 2 -- Welfare Commitment Evaluation}

The candidate must demonstrate a sustained, good-faith commitment to one or more of the following:
\begin{itemize}
\item animal welfare,
\item human welfare,
\item responsible animal stewardship.
\end{itemize}

Evaluation is based on documented conduct, not self-asserted values or affinity.

Acceptable evidence may include, without limitation:
\begin{itemize}
\item long-term primary dog ownership or caregiving history,
\item foster, rescue, or shelter involvement,
\item veterinary, behavioral, training, or welfare-adjacent professional work,
\item governance, financial, or operational support of welfare organizations,
\item sustained volunteer or advocacy work involving animals or vulnerable people,
\item other documented activities demonstrating responsibility, restraint, and care.
\end{itemize}

No single pathway is required.

\noindent
\texttt{IF evidence demonstrates sustained engagement, THEN proceed.} \\
\texttt{IF evidence is episodic, superficial, or purely declarative, THEN terminate vetting.}

\subsection{Stage 3 -- Disqualification Screening}

The candidate must attest, and the organization must reasonably confirm, that none of the following apply:
\begin{itemize}
\item conviction for animal cruelty, animal abuse, or animal neglect,
\item substantiated judicial or regulatory finding for animal cruelty, abuse, or neglect,
\item conviction or substantiated finding for cruelty toward humans.
\end{itemize}

Where uncertainty exists, classification defaults to disqualification until resolved.

Reasonable confirmation does not require exhaustive investigation and may rely on good-faith attestations, public records, and information reasonably available at the time.


\noindent
\texttt{IF any disqualifying record exists, THEN terminate vetting.} \\
\texttt{ELSE proceed to Stage 4.}

\subsection{Stage 4 -- Continuing-Condition Acknowledgment}

The candidate must acknowledge in writing that:
\begin{itemize}
\item Governing Body membership is conditional on continued compliance with eligibility and disqualification requirements,
\item discovery of disqualifying conduct triggers mandatory review and removal,
\item omission or concealment of relevant information may itself constitute cause for removal.
\end{itemize}

\section{Documentation Requirements}

For each vetted candidate, the following must be retained in governance records:
\begin{itemize}
\item written disclosure statement,
\item welfare-commitment evidence summary,
\item disqualification attestation,
\item names of reviewers,
\item date of final eligibility determination.
\end{itemize}

All records must be reviewable upon request.

\section{Prohibited Practices}

This vetting process must not:
\begin{itemize}
\item rely on subjective affinity or narrative alone,
\item impose dog ownership as a requirement,
\item evaluate ideological alignment unrelated to welfare,
\item apply standards selectively or inconsistently,
\item create informal exceptions or waivers.
\end{itemize}

\section{Relationship to Removal and Classification}

This process:
\begin{itemize}
\item does not limit Governing Body removal authority,
\item does not preempt Classification Panel review,
\item does not shield a member from consequences of later findings.
\end{itemize}

Discovery of disqualifying conduct after appointment must be handled under the Governing Body removal provisions and the Classification, Arbitration, and Accountability framework.

\section{Review and Revision}

This vetting process may be revised to improve clarity or reflect new evidence types, provided that revisions:
\begin{itemize}
\item remain strictly derivative of the Constitution,
\item do not narrow constitutional disqualifications,
\item do not add new eligibility constraints,
\item do not create discretion beyond evaluation.
\end{itemize}

\end{document}
