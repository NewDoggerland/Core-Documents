\documentclass[11pt]{article}
\usepackage{ndstyle}

\title{New Doggerland - Operational Manual v0.6.0}
\author{}
\date{}
\begin{document}
\maketitle
\tableofcontents
\newpage


\section{How to use this manual}

This manual is written for execution. It is organized so you can:
\begin{itemize}
\item find the relevant procedure quickly,
\item understand what must be done vs.\ what may be adapted, and
\item know when to escalate.
\end{itemize}

\subsection{Two-pass reading}

\begin{enumerate}
\item \textbf{First pass (15--20 minutes):} skim section headings to learn where things live.
\item \textbf{Second pass (as-needed):} use the section that matches the task you are doing; do not read linearly unless onboarding.
\end{enumerate}

\subsection{Rule of thumb for escalation}

Escalate when any of the following are true:
\begin{itemize}
\item the decision is materially irreversible,
\item the decision materially affects dog welfare, human dignity, or long-horizon stewardship,
\item the decision creates a precedent others will rely on,
\item you cannot explain the choice in a way that would satisfy a good-faith reviewer.
\end{itemize}

\noindent
When in doubt: choose the \textbf{most welfare-preserving} and \textbf{least power-expanding} option, then request review.


\section{OM-0. General Operating Rules}

\begin{itemize}
\item \texttt{IF an operational instruction conflicts with the Constitution, THEN the instruction is void and must not be followed.}
\item \texttt{IF an operational instruction conflicts with law, THEN the instruction must be escalated to Governing Body for correction.}
\item \texttt{IF an operational gap is discovered, THEN staff must halt discretionary action and escalate for guidance, EXCEPT THAT: IF the affected action is routine, reversible, and within the actor’s role scope, THEN the actor may implement a temporary bounded control ONLY IF the control creates no lasting obligation, transfers or risks no assets, is reversible without Governing Body approval, is logged at time of action with a GAP-BRIDGE reason code and an explicit expiration time not exceeding 72 hours, AND the Governing Body is notified for review as soon as feasible.}

\item \texttt{For purposes of this section, an operational gap exists only when no existing instruction, role norm, or prior recorded decision pathway reasonably applies to the action as executed.}

\item \texttt{This gap-bridge cannot be used to authorize contracts, spending, partner commitments, staff discipline, access revocation, policy changes, or any action that would be a Material Action under the Constitution as defined in Governance Schedule A (10.10).}

\item \texttt{IF a GAP-BRIDGE is used for the same operational gap or a substantially similar action more than once within any rolling 30-day period, THEN the Governing Body must place the gap on the next regular meeting agenda for corrective instruction, AND further GAP-BRIDGE use for that gap is prohibited unless Governing Body expressly authorizes continuation in writing.}

\item \texttt{IF an operational gap is discovered AND stopping work would cause imminent harm to dog welfare or human dignity,
THEN invoke Emergency Invocation (OM-11) only to implement the minimum temporary control needed to prevent that harm,
AND record (1) the gap, (2) the harm risk, (3) the temporary control, (4) who authorized it, and (5) an expiration time,
AND notify Governing Body for review as soon as feasible,
ELSE halt the affected action until Governing Body issues guidance.
}
\item \texttt{OM-0 GAP-BRIDGE and OM-1 PROVISIONAL-AUTHORITY are two labels for the same temporary control class: PROVISIONAL CONTROL. IF either pathway is invoked, THEN all prohibitions, logging requirements, expiration limits, and non-precedent rules from BOTH sections apply cumulatively, and the strictest constraint controls.}

\end{itemize}

\section{OM-1. Authority \& Decision Escalation}

\begin{itemize}
\item \texttt{IF a decision materially affects welfare, safety, dignity, assets, or land, THEN documented approval by the Governing Body is required.}

\item \texttt{Material Action thresholds and escalation triggers are defined exclusively in Governance Schedule A (10.10).}

\item \texttt{IF a decision is routine and reversible, THEN designated staff may act within their role scope.}

\item \texttt{IF authority is unclear, THEN the decision must not proceed until clarified, EXCEPT THAT: IF the decision is routine, reversible, and within the actor’s role scope, THEN the actor may proceed provisionally ONLY IF the actor creates a log stub at time of action with a PROVISIONAL-AUTHORITY reason code, states the narrow intended scope, and sets an expiration time not exceeding 72 hours; AND IF any reasonable disagreement arises that the action is materially affecting OR if the action proves not reversible without further authorization, THEN the provisional action must stop and be escalated immediately for Governing Body review; AND this provisional pathway cannot be used to create obligations, commit funds, discipline staff, revoke access, or establish precedent.}

\item \texttt{IF PROVISIONAL-AUTHORITY is invoked for the same decision category more than once within any rolling 30-day period, THEN the actor must escalate for Governing Body clarification before any further provisional action in that category.}

\item \texttt{For purposes of this section, a decision category is defined by the nature of the authority ambiguity and the class of impact, not by superficial differences in fact pattern or execution context.}

\item \texttt{Authority is not unclear solely because an action has not been previously executed, documented, or reviewed, provided it is routine, reversible, and within established role scope.}


\end{itemize}

\section{OM-2. Stewardship Compliance}

\begin{itemize}
\item \texttt{IF a proposed action prioritizes short-term gain over long-term stewardship, THEN the action must not proceed.}
\item \texttt{IF a proposal risks degradation of welfare, safety, or dignity, THEN it must be rejected or redesigned.}
\item \texttt{IF maintenance capacity is insufficient, THEN expansion or acquisition must pause.}
\end{itemize}

\section{OM-3. Asset \& Infrastructure Management}

\begin{itemize}
\item \texttt{IF an asset cannot be reasonably maintained, THEN it must not be acquired or must be retired.}
\item \texttt{IF an asset degrades welfare or safety, THEN it must be repaired, replaced, or removed.}
\item \texttt{IF durable and repairable options exist at reasonable cost, THEN disposable options must not be selected.}
\end{itemize}

\section{OM-4. Partner \& Vendor Engagement}

\begin{itemize}
\item \texttt{IF a partner relationship involves extraction, exploitation, or misalignment, THEN it must not be entered.}
\item \texttt{IF alignment is uncertain, THEN provisional engagement requires Governing Body review.}
\item \texttt{IF a partner violates ND constraints, THEN the relationship must be suspended pending review.}
\end{itemize}

\section{OM-5. Staff Conduct \& Dignity}

\begin{itemize}
\item \texttt{IF staff act in good faith and make errors, THEN correction and support are required, not punishment.}
\item \texttt{IF staff act in bad faith, conceal harm, or retaliate, THEN disciplinary action is mandatory.}
\item \texttt{IF dignity is compromised, THEN operations must stop until the issue is addressed.}
\end{itemize}

\section{OM-6. Safety \& Hazard Response}

\begin{itemize}
\item \texttt{IF a hazard is identified, THEN it must be marked, logged, and communicated.}
\item \texttt{IF a hazard poses immediate risk, THEN access must be restricted.}
\item \texttt{IF conditions return to a safe state, THEN restrictions may be lifted.}
\end{itemize}

\section{OM-7. Overrides \& Exceptions}

\begin{itemize}
\item \texttt{IF an override is required to prevent immediate harm, THEN it must be documented with reason, scope, and duration.}
\item \texttt{IF an override is undocumented, THEN it is invalid.}
\item \texttt{IF an override exceeds necessity, THEN it must be reversed.}
\item \texttt{IF an override occurs, THEN it must be classified under the Constitution’s Classification, Arbitration, and Accountability provisions.}
\end{itemize}

\section{OM-8. Transparency \& Logging}

\begin{itemize}
\item \texttt{IF a material action occurs, THEN it must be logged.}
\item \texttt{IF an action cannot be logged in the primary system, THEN it must not occur, EXCEPT THAT: IF the action is routine, reversible, and within role scope, THEN the action may proceed ONLY IF the actor creates an offline log stub at time of action containing timestamp, actor, scope, reason code, and invocation basis, and the stub is stored in a tamper-evident manner designated by Governance Schedule D (10.40).}

\item \texttt{IF an offline stub is used, THEN the actor must enter the stub into the primary system within 72 hours of restored logging capability, as required by Governance Schedule D (10.40).}

\item \texttt{IF offline stub usage recurs beyond defined norms OR any required stub field is missing, THEN automatic review is triggered under OM-11 or the Constitution’s Classification, Arbitration, and Accountability provisions as applicable.}

\item \texttt{IF records are incomplete, THEN remediation is required.}
\item \texttt{IF an emergency requires immediate action, THEN action may proceed without prior logging solely to prevent imminent harm.}
\item \texttt{IF logging is delayed beyond the emergency window, THEN the action must undergo classification review under OM-11.}
\item \texttt{Logging is satisfied at time of action by creation of a minimal log stub containing timestamp, actor, scope, reason code, and invocation basis. Full narrative logging may be completed later as permitted elsewhere in this Manual.}
\item \texttt{A valid emergency log stub must reference a pre-defined emergency category code and affected registry objects. Free-text-only stubs are invalid.}

\end{itemize}

\section{OM-9. Failure Handling}

\begin{itemize}
\item \texttt{IF failure occurs without bad faith, THEN transparency and correction are required.}
\item \texttt{IF failure involves concealment or intent to harm, THEN escalation is mandatory.}
\item \texttt{IF learning can prevent recurrence, THEN procedures must be updated.}
\end{itemize}

\section{OM-10. Review \& Evolution}

\begin{itemize}
\item \texttt{IF an operational rule causes harm or friction, THEN it must be reviewed.}
\item \texttt{IF improvements are identified, THEN revisions may be proposed.}
\item \texttt{IF revisions conflict with the Constitution, THEN they must not be adopted.}
\end{itemize}

\section{OM-11. Emergency Authority}

\subsection{OM-11.A Invocation Rules}

\texttt{IF immediate action is required and normal authorization is infeasible, THEN staff may act narrowly to prevent harm.}

\texttt{IF emergency action occurs, THEN the action must be logged within the emergency window or undergo classification review under the Constitution’s Classification, Arbitration, and Accountability provisions.}


\texttt{Emergency authority may only affect assets, systems, or persons explicitly listed in the Emergency Scope Registry as defined in the Constitution.
 Actions outside the registry are void.}

\texttt{Each emergency invocation authorizes either (A) one discrete action, OR (B) one tightly-related response bundle consisting only of actions necessary to control a single emergency condition.}

\texttt{Emergency invocation does not establish precedent and must not be cited as authorization for future actions outside the specific emergency condition documented.}

\texttt{IF a response bundle is used, THEN the actor must (1) list each bundled action in the emergency log stub, (2) state the single emergency condition being controlled, (3) state the stop condition, and (4) set an expiration time not exceeding 4 hours.}

\texttt{IF the emergency condition persists beyond the expiration time OR the scope expands to a different emergency condition, THEN a new emergency invocation and new log stub are required.}

\texttt{A response bundle must not include contracts, partner commitments, staff discipline, access revocation, policy changes, or any action that is irreversible without Governing Body approval.}


\subsection{OM-11.B Proportionality Test}

Emergency action is narrowly scoped ONLY IF:
\begin{itemize}
\item it directly prevents imminent harm
\item no less invasive alternative is available
\item duration is limited
\item no unrelated risks are introduced
\item Emergency authority must not be used for actions that are irreversible without Governing Body approval.
\end{itemize}

\texttt{IF any criterion is not met, THEN classification review is mandatory.}

\subsection{OM-11.C Rolling Invocation Trigger}

\texttt{IF emergency authority is invoked more than once by the same individual within a rolling 90-day window, THEN an automatic pattern review must occur.}

\texttt{IF an individual invokes emergency authority, THEN any additional invocation by that individual within the next 24 hours automatically triggers a mandatory pattern review flag, but does not prohibit invocation when needed to prevent imminent harm.}

Pattern review must not presume misconduct and must default to Mistake classification unless evidence supports Violation or Abuse under the Constitution’s Classification, Arbitration, and Accountability provisions.


\subsection{OM-11.D Audit Hooks}

\texttt{IF emergency authority is invoked, THEN the system must:}
\begin{itemize}
\item \texttt{Timestamp the action}
\item \texttt{Flag the actor’s account}
\item \texttt{Schedule mandatory log entry within 72 hours}
\item \texttt{Notify Governing Body and Audit roles}
\end{itemize}
Emergency invocations must be logged and made available for external arbiter or audit observer review.
Automatic notification to an external arbiter or audit observer is required only if one or more of the following conditions are met:
\begin{itemize}
\item the emergency action is not fully reversed within its declared expiration window
\item the emergency action materially affects a person, asset, or domain beyond the immediate emergency condition
\item the same actor invokes emergency authority more than once within a rolling 90-day window
\item any required log element is missing, inconsistent, or delayed beyond permitted limits
\end{itemize}



\subsection{OM-11.E Enforcement}

\begin{itemize}
\item All emergency invocations automatically start a review clock.
\item \texttt{IF the actor is incapacitated, displaced, detained, or under coercion, THEN the logging deadline is tolled until the incapacity ends.}
\item \texttt{IF incapacity tolling is invoked, THEN the actor must complete logging within 72 hours of restored capacity.}
\item \texttt{IF incapacity tolling exceeds 14 days, THEN automatic Classification Panel review is triggered.}
\item \texttt{IF incapacity tolling is claimed without reasonable basis or supporting evidence, THEN the action escalates directly to Violation review.}
\texttt{IF a Classification Panel review is triggered, THEN the invoking actor’s emergency authority is automatically suspended until review closure.}
\item \texttt{Expiration of the review clock without documented closure requires automatic human review by the Classification Panel or its delegate.}
\item \texttt{No emergency action may bypass quorum, term-validity, or classification safeguards required by the Constitution.}
\end{itemize}

\subsection{OM-11.F Emergency Frequency Monitoring}

\texttt{IF emergency authority is invoked beyond peer norms within a rolling period, THEN the system must generate a comparative analytics report and initiate mandatory Classification Panel review.}

\subsection{OM-11.G Load Shedding and Triage}

\texttt{IF review volume exceeds capacity thresholds, THEN the Classification Panel must prioritize cases by demonstrated or potential harm. Deferred cases remain logged and reviewable.}


\appendix
\section{OM-A. Pre-Registration Operating Kernel (Day-1)}

\noindent
\textbf{Status:} Binding during pre-registration phase only \\
\textbf{Authority:} Subordinate to the Constitution and Consolidated Governance Document \\
\textbf{Expiration:} This section expires automatically upon formal organizational registration or activation of a Governing Body, whichever occurs first.

\subsection{OM-A.1 Scope and Function}

This section defines a temporary execution kernel applicable only during the pre-registration phase.

This section creates no governance authority, confers no voting power, and does not modify any role defined in higher-order documents.

\subsection{OM-A.2 Role-Limited Authority (Day-1)}

\texttt{IF the Founder is acting in an operational capacity, THEN the Founder may execute only routine, reversible, role-scoped actions necessary for care, safety, or maintenance.}

\texttt{IF Emergency Authority is invoked, THEN it must comply fully with the Emergency Invocation and Emergency Scope Registry provisions.}

\texttt{IF an action would create an obligation, contract, disciplinary effect, exclusion, or irreversible commitment, THEN the action must not be taken.}

\texttt{IF any other person is present in a helper, volunteer, or contractor capacity, THEN that person may act only under explicit role-scoped instruction or Emergency Authority.}

\texttt{IF authority is unclear, THEN the action must halt unless it is routine, reversible, and logged as provisional under GAP-BRIDGE rules.}

\subsection{OM-A.3 Mandatory Logging Requirements}

\texttt{IF any of the following occur, THEN the action must be logged at the time of action or immediately upon stabilization:}
\begin{itemize}
\item Emergency Authority invocation
\item Provisional or GAP-BRIDGE control use
\item Any action affecting dog welfare
\item Any action affecting human dignity
\item Any access restriction
\item Any safety condition
\item Any asset use beyond trivial consumption
\end{itemize}

Minimum required log fields during pre-registration:
\begin{itemize}
\item Timestamp
\item Actor
\item Bounded factual description
\item Reason code (EMERGENCY / PROVISIONAL / ROUTINE)
\item Explicit expiration, if applicable
\end{itemize}

\texttt{IF immediate logging is not possible due to active harm prevention, THEN logging must occur as soon as the condition stabilizes and must state the delay reason.}

\texttt{Logs must be reviewed periodically to identify cumulative effects that may not be evident from individual actions.}


\subsection{OM-A.4 Mandatory Escalation Conditions}

\texttt{IF an action is not reversible, THEN it must be escalated.}

\texttt{IF an action creates precedent, THEN it must be escalated.}

\texttt{IF an action affects housing, access, discipline, or exclusion, THEN it must be escalated.}

\texttt{IF the actor cannot explain the action to a neutral reviewer without defensiveness, THEN the action must be escalated.}

\texttt{IF no escalation body exists, THEN the actor must halt unless imminent harm exists, in which case the GAP-BRIDGE pathway must be used with explicit expiration and notification.}

\subsection{OM-A.5 Absolute Prohibitions}

The following actions are prohibited at all times, including during emergencies:

\texttt{Unlogged emergency actions are prohibited.}

\texttt{Informal bans, evictions, or access revocations are prohibited.}

\texttt{Compensation-by-sacrifice expectations are prohibited.}

\texttt{Consensus claims without a record are prohibited.}

\texttt{Silent or informal overrides are prohibited.}

\texttt{IF an action is procedurally prohibited, THEN it must not be taken regardless of perceived moral justification.}

\subsection{OM-A.6 Four-Hour Emergency Limitation}

\texttt{IF Emergency Authority is invoked, THEN actions must be narrowly scoped to a single harm condition.}

\texttt{IF four hours elapse after Emergency Authority invocation, THEN either the emergency must be declared resolved or the matter must transition to governance-level handling.}

\texttt{Emergency Authority must not be used to normalize unresolved conditions.}


\section{OM-B. Pre-Registration Pilot Guidance (First 30 Days)}

\noindent
\textbf{Status:} Non-binding execution guidance \\
\textbf{Authority:} Subordinate to Operational Manual \\
\textbf{Function:} Planning and classification aid only

\subsection{OM-B.1 Actual Roles (No Assumed Authority)}

Recognized roles during pre-registration are limited to:
\begin{itemize}
\item Founder acting as operational staff
\item Helper or volunteer with no independent discretion
\item Dog groomer (if facilities are present)
\item External advisors with no authority
\end{itemize}

No additional roles or committees are implied.

\subsection{OM-B.2 Decision Classification Table}

The following table is a classification aid only. It does not grant permission.

\begin{center}
\begin{tabular}{|l|l|l|}
\hline
Decision & Routine and Reversible & Notes \\
\hline
Temporary dog intake & Yes (bounded) & Explicit duration cap required \\
Emergency veterinary care & Yes & Emergency scope only \\
Food sourcing & Yes & No contracts permitted \\
Small equipment purchase & Yes & Subject to spending cap defined in Governance Schedule A (10.10) \\
Visitor presence & No & Requires explicit rule \\
Overnight guest & No & Default deny \\
Dog isolation & Yes & Must be logged \\
Access restriction & Emergency only & Never informal \\
Housing use & No & Governance-level \\
Media contact & No & Must not improvise \\
\hline
\end{tabular}
\end{center}

\texttt{IF a decision is not explicitly classified, THEN the decision must not be executed except under Emergency Invocation.}

\subsection{OM-B.3 Carrying-Capacity Tripwires}

\texttt{IF any of the following conditions occur, THEN expansion actions must pause immediately:}
\begin{itemize}
\item Any dog lacks daily care coverage
\item Maintenance backlog exceeds seven days
\item Logging falls behind more than seventy-two hours
\item Emergency Authority is invoked twice within thirty days
\end{itemize}

\subsection{OM-B.4 Early-Stage Logging Mechanism}

During pre-registration, the following logging mechanism is permitted:
\begin{itemize}
\item One bound logbook with numbered pages
\item One ink color only
\item No page removal
\item Index maintained at front
\end{itemize}

This allowance expires upon digital system activation.

\texttt{IF the bound-logbook mechanism remains the sole logging system beyond ninety (90) days of operation, THEN the acting operator must document the reason for delay and place digital logging transition on the next governance or review agenda.}

\subsection{OM-B.5 Weekly Self-Review Requirement}

\texttt{IF operating during pre-registration, THEN once per week the acting operator must:}
\begin{itemize}
\item Review all logs
\item Flag any discomfort or ambiguity
\item Record one sentence addressing potential external review concern
\end{itemize}

\texttt{Failure to perform weekly review constitutes operational degradation.}




\end{document}
