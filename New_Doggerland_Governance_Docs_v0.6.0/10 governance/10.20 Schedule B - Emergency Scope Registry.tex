
\documentclass[11pt]{article}
\usepackage{ndstyle}
\setcounter{secnumdepth}{2}
\setcounter{tocdepth}{2}

\title{New Doggerland \\ Governance Schedules \\ Schedule B - Emergency Scope Registry}
\author{}
\date{}

\begin{document}
\maketitle
\tableofcontents
\newpage

\noindent
\textbf{Status:} Binding Governance Schedule \\
\textbf{Authority:} Adopted by the Governing Body pursuant to the Consolidated Governance Document \\
\textbf{Scope:} Defines the exclusive objects, systems, and persons within which emergency authority may be exercised \\
\textbf{Precedence:} Subordinate to the Consolidated Governance Document; controlling where referenced \\
\textbf{Mutability:} Revisable only by Extraordinary Process

\section{Purpose}

This Schedule enumerates the exclusive scope of persons, animals, assets, spaces, and systems that may be affected by emergency authority as defined in the Consolidated Governance Document and the Emergency Scope Registry provisions therein.

No emergency authority may be exercised outside the scope explicitly defined in this Schedule.

\section{General Constraints}

Emergency authority:
\begin{itemize}
\item exists solely to prevent imminent harm,
\item may be exercised only within the scope objects enumerated in this Schedule,
\item must comply with all trigger standards, stop conditions, and expiration limits defined in the Emergency Scope Registry,
\item must be logged as a Material Action using an emergency log stub.
\end{itemize}

Silence in this Schedule does not confer authority.

\section{In-Scope Persons}

Emergency authority may affect the following person classes solely for purposes of immediate safety and harm prevention:

\begin{itemize}
\item \texttt{PERSON.GUEST}
\item \texttt{PERSON.STAFF}
\item \texttt{PERSON.VOLUNTEER}
\item \texttt{PERSON.PARTICIPANT}
\end{itemize}

Emergency authority must not:
\begin{itemize}
\item revoke rights,
\item impose discipline,
\item authorize eviction, banning, or trespass,
\item create long-term restrictions on access.
\end{itemize}

\section{In-Scope Animals}

Emergency authority may affect the following animal classes:

\begin{itemize}
\item \texttt{ANIMAL.DOG}
\item \texttt{ANIMAL.OTHER}
\end{itemize}

Emergency authority must not authorize irreversible medical decisions, euthanasia, or experimental treatment.

\section{In-Scope Physical Spaces}

Emergency authority may affect access to or use of the following space classes:

\begin{itemize}
\item \texttt{SPACE.BUILDING}
\item \texttt{SPACE.ROOM}
\item \texttt{SPACE.CORRIDOR}
\item \texttt{SPACE.TRAIL}
\item \texttt{SPACE.FIELD}
\item \texttt{SPACE.WATER\_FEATURE}
\item \texttt{SPACE.PARKING}
\end{itemize}

Emergency authority may only restrict the minimum spatial scope necessary to prevent harm.

\section{In-Scope Assets and Utilities}

Emergency authority may affect the following asset classes:

\begin{itemize}
\item \texttt{ASSET.UTILITY.GAS}
\item \texttt{ASSET.UTILITY.ELECTRIC}
\item \texttt{ASSET.UTILITY.WATER}
\item \texttt{ASSET.UTILITY.HVAC}
\item \texttt{ASSET.SAFETY.FIRE\_EXT}
\item \texttt{ASSET.SAFETY.AED}
\item \texttt{ASSET.SAFETY.FIRST\_AID}
\item \texttt{ASSET.SAFETY.BARRICADE\_KIT}
\item \texttt{ASSET.SAFETY.SIGNAGE}
\end{itemize}

Emergency authority must not authorize capital modification, replacement, or procurement beyond pre-approved agreements.

\section{In-Scope Systems}

Emergency authority may affect the following systems:

\begin{itemize}
\item \texttt{SYSTEM.ACCESS\_CONTROL}
\item \texttt{SYSTEM.LOGGING\_PRIMARY}
\item \texttt{SYSTEM.LOGGING\_OFFLINE}
\end{itemize}

System access under emergency authority must:
\begin{itemize}
\item be time-limited,
\item preserve auditability,
\item revert automatically upon stop condition.
\end{itemize}

\section{Explicit Exclusions}

Emergency authority must not be exercised with respect to:

\begin{itemize}
\item financial accounts or banking systems,
\item contracts or vendor commitments,
\item compensation, payroll, or benefits,
\item data systems not explicitly listed as in-scope,
\item governance records other than emergency log stubs.
\end{itemize}

\section{Interpretive Rule}

This Schedule must be interpreted narrowly. Where ambiguity exists, interpretation must favor:
\begin{itemize}
\item harm minimization,
\item dignity preservation,
\item the smallest effective scope.
\end{itemize}

\section{Amendment and Review}

This Schedule may be amended only through an Extraordinary Process as defined in the Consolidated Governance Document.

All amendments must:
\begin{itemize}
\item be adopted by recorded vote,
\item be versioned,
\item specify additions, removals, or modifications explicitly.
\end{itemize}

\end{document}
