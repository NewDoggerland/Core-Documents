\documentclass[11pt]{article}
\usepackage{ndstyle}

\title{New Doggerland - Operational Manual Companion Document v0.6.1}
\author{}
\date{}

\begin{document}
\maketitle
\tableofcontents
\newpage

\noindent
\textbf{Status:} Non-binding explanatory companion \\
\textbf{Authority:} Subordinate to ``New Doggerland - Operational Manual v0.6.1'' and, above it, to ``New Doggerland - Consolidated Governance Document v0.6.1'' \\
\textbf{Purpose:} Explain intentions and mechanisms without creating new obligations

\section{How to use this companion}

This companion is here to make the operational manual easier to live with:
\begin{itemize}
\item it explains \textbf{why} procedures exist,
\item it clarifies \textbf{how to apply judgment} without drifting out of bounds,
\item it gives a shared language for \textbf{tradeoffs}, \textbf{exceptions}, and \textbf{review}.
\end{itemize}

\subsection{If you are onboarding}

Read, in order:
\begin{enumerate}
\item this section,
\item the sections that define \textbf{discretion limits} and \textbf{escalation},
\item only then the manual sections you will actively use in your role.
\end{enumerate}

\subsection{What this companion cannot do}

\noindent
This companion does not override the operational manual or the consolidated governance document. Where there is any conflict, the higher-authority document controls.

\section{What this document is and is not}

\subsection{What this document is}

This document is a readable explanation of how the Operational Manual is designed to function in practice. It is written to help:
\begin{itemize}
\item staff,
\item supervisors,
\item partner organizations,
\item auditors, and
\item future stewards
\end{itemize}
understand the operational logic quickly and accurately.

\subsection{What this document is not}

This document is not:
\begin{itemize}
\item a source of authority,
\item policy,
\item a substitute for the Operational Manual,
\item a mechanism to reinterpret rules, or
\item a permission to deviate from the Operational Manual.
\end{itemize}

If there is any difference between this companion and the Operational Manual, the Operational Manual controls. If there is any difference between the Operational Manual and the Consolidated Governance Document, the Consolidated Governance Document controls.

\section{Design intent in one paragraph}

The Operational Manual exists to convert New Doggerland’s governance constraints into executable operational behavior. It assumes that front-line work encounters ambiguity, missing instructions, system outages, urgent conditions, and human error. It therefore encodes narrow default rules, escalation triggers, and logging requirements so that routine work can continue safely, while non-routine or materially affecting actions are forced into attributable, reviewable pathways.

\section{How to read the Operational Manual}

\subsection{IF / THEN logic is the interface}

The Operational Manual is written as IF / THEN logic because staff must be able to execute rules under stress without interpretive debate. The goal is mechanical compliance: identify the trigger, perform the required action, and log what the rule requires.

\subsection{The Manual is subordinate and cannot create new authority}

The Manual cannot create new powers, override prohibitions, or expand emergency authority beyond what the Governance Document permits. Where a rule requires escalation, the escalation requirement exists specifically to prevent staff from being placed in the role of governance.

\section{OM-0 General Operating Rules}

\subsection{Conflict handling}

OM-0 establishes the primary conflict rules:
\begin{itemize}
\item If an operational instruction conflicts with the Constitution, it is void.
\item If an operational instruction conflicts with law, it must be escalated for correction.
\end{itemize}

This prevents a common failure mode where staff follow an internal checklist that is wrong but ``in writing.''

\subsection{Operational gaps and the GAP-BRIDGE / PROVISIONAL CONTROL mechanism}

OM-0 defines an operational gap narrowly: it exists only when no existing instruction, role norm, or prior recorded decision pathway reasonably applies to the action as executed.

When a gap exists, the default is to halt discretionary action and escalate for guidance. The only exception is the GAP-BRIDGE pathway, which is a tightly constrained temporary control permitted only when the action is routine, reversible, within role scope, creates no lasting obligation, risks no assets, and is logged with a reason code and short expiration.

Key design intent:
\begin{itemize}
\item Allow limited continuity for small reversible actions.
\item Prevent the gap mechanism from becoming a shadow authority channel.
\end{itemize}

OM-0 also forces corrective governance action when a repeated gap appears. If the same gap is bridged more than once in a rolling 30-day period, governance must address it and further gap-bridge use is prohibited absent written continuation.

\subsection{Imminent harm exception routes to Emergency Invocation}

If an operational gap exists and stopping work would cause imminent harm to dog welfare or human dignity, OM-0 routes the actor to Emergency Invocation (OM-11) with a minimum-control requirement, explicit record fields, and a short expiration.

This is intended to ensure that harm-prevention is possible without turning ``gaps'' into permissions.

\subsection{GAP-BRIDGE and PROVISIONAL-AUTHORITY are unified}

OM-0 explicitly states that OM-0 GAP-BRIDGE and OM-1 PROVISIONAL-AUTHORITY are two labels for the same temporary control class (PROVISIONAL CONTROL) and that the strictest constraints apply cumulatively.

This is intended to prevent label-shopping.

\section{OM-1 Authority and Decision Escalation}

OM-1 operationalizes the basic decision boundary:
\begin{itemize}
\item Materially affecting decisions require documented approval by the governing authority.
\item Routine and reversible decisions may be executed within role scope.
\end{itemize}

OM-1 then provides a narrow provisional pathway when authority is unclear. The mechanism requires contemporaneous logging, narrow scope, short expiration, and immediate escalation if reversibility fails or reasonable disagreement arises about materiality. It also prohibits use for obligations, funds, discipline, access revocation, or precedent.

OM-1 further defines a ``decision category'' as the nature of the authority ambiguity and class of impact, preventing superficial variations from being used to evade the 30-day repetition trigger.

\section{OM-2 Stewardship Compliance}

OM-2 turns stewardship principles into stop rules:
\begin{itemize}
\item Short-term gain cannot override long-term stewardship.
\item Proposals that risk welfare, safety, or dignity must be rejected or redesigned.
\item Expansion pauses when maintenance capacity is insufficient.
\end{itemize}

This exists to prevent growth from outpacing the organization’s ability to maintain humane, safe operations.

\section{OM-3 Asset and Infrastructure Management}

OM-3 imposes acquisition and maintenance discipline:
\begin{itemize}
\item Do not acquire what cannot be maintained.
\item Repair, replace, or remove assets that degrade welfare or safety.
\item Prefer durable and repairable options over disposable ones when reasonable.
\end{itemize}

The design intent is ``nonprofits can’t afford to buy cheap things'' translated into operational stop conditions.

\section{OM-4 Partner and Vendor Engagement}

OM-4 is designed to prevent extraction and misalignment entering through partnerships:
\begin{itemize}
\item Do not enter exploitative or misaligned relationships.
\item If alignment is uncertain, provisional engagement requires governance review.
\item If a partner violates ND constraints, suspend pending review.
\end{itemize}

This prevents partners from becoming a backdoor governance channel.

\section{OM-5 Staff Conduct and Dignity}

OM-5 separates error from misconduct:
\begin{itemize}
\item Good-faith errors require correction and support, not punishment.
\item Bad faith, concealment, harm, or retaliation requires mandatory discipline.
\item If dignity is compromised, operations stop until addressed.
\end{itemize}

The intent is consistent with the Governance Document’s classification logic and anti-retaliation posture.

\section{OM-6 Safety and Hazard Response}

OM-6 operationalizes hazard handling:
\begin{itemize}
\item Identify, mark, log, and communicate hazards.
\item Restrict access when hazards pose immediate risk.
\item Lift restrictions when conditions return to a safe state.
\end{itemize}

This provides consistent field behavior that can be audited.

\section{OM-7 Overrides and Exceptions}

OM-7 treats overrides as exceptional and reviewable:
\begin{itemize}
\item Overrides are permitted only to prevent immediate harm and must be documented with reason, scope, and duration.
\item Undocumented overrides are invalid.
\item Excessive overrides must be reversed.
\item Overrides must be classified under the Governance Document’s classification provisions.
\end{itemize}

This exists to prevent ``temporary'' exceptions from silently becoming permanent.

\section{OM-8 Transparency and Logging}

OM-8 treats logging as a condition of legitimacy:
\begin{itemize}
\item Material actions must be logged.
\item If primary logging is unavailable, actions generally must not occur.
\end{itemize}

The exception allows routine, reversible, role-scoped actions to proceed with an offline stub containing required fields, stored tamper-evidently, and entered into the primary system within 72 hours of restored capability.

The design intent is:
\begin{itemize}
\item preserve continuity for small safe actions during outages,
\item keep the audit spine intact,
\item force review when offline logging deviates from norms or is incomplete.
\end{itemize}

OM-8 also defines minimum stub content and requires emergency stubs to reference pre-defined emergency category codes and registry objects, rejecting free-text-only stubs.

\section{OM-9 Failure Handling}

OM-9 enforces the response pattern:
\begin{itemize}
\item Good-faith failure requires transparency and correction.
\item Concealment or intent to harm requires mandatory escalation.
\item Procedures must update when learning can prevent recurrence.
\end{itemize}

\section{OM-10 Review and Evolution}

OM-10 is the improvement loop:
\begin{itemize}
\item Harm or friction triggers review.
\item Improvements may be proposed.
\item Conflicts with the Constitution bar adoption.
\end{itemize}

\section{OM-11 Emergency Authority}

\subsection{Invocation rules}

OM-11 defines emergency authority as narrow action to prevent harm when normal authorization is infeasible, within the Emergency Scope Registry, and with explicit non-precedent constraints. It distinguishes between one discrete action and one tightly-related response bundle, with listing requirements, stop conditions, and short expirations.

It explicitly excludes contracts, partner commitments, discipline, access revocation, policy changes, and irreversible actions without Governing Body approval.

\subsection{Proportionality test}

OM-11 imposes a proportionality test to ensure emergency actions remain narrowly scoped and require classification review if criteria are not met.

\subsection{Rolling invocation triggers and pattern review}

OM-11 triggers automatic review when the same individual invokes emergency authority more than once in a rolling window and flags repeat invocations within 24 hours. Pattern review does not presume misconduct and defaults to Mistake unless evidence supports Violation or Abuse under the Governance Document.

\subsection{Audit hooks and enforcement}

OM-11 specifies system behaviors (timestamping, flagging, scheduling mandatory log entry, notifications) and defines when external arbiter or audit observer notification is required based on reversibility, scope expansion, repeat invocation, or logging defects.

It also defines tolling rules for incapacity and escalation triggers if tolling is abused or exceeds limits, including automatic emergency authority suspension pending review when a Classification Panel review is triggered.

\subsection{Frequency monitoring and load shedding}

OM-11 includes comparative analytics reporting for outlier emergency use and a triage rule when review volume exceeds capacity, prioritizing by harm while keeping deferred cases logged and reviewable.

\section{Companion document maintenance rule}

This companion document may be updated for clarity and readability, but must remain strictly derived from the Operational Manual. Any update that could be interpreted as creating new obligations must be rejected.

\end{document}
