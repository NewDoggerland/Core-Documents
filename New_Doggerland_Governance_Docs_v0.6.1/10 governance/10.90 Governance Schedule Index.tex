\documentclass[11pt]{article}
\usepackage{ndstyle}
\setcounter{secnumdepth}{2}
\setcounter{tocdepth}{2}

\title{New Doggerland \\ Governance Schedule Index}
\author{}
\date{}

\begin{document}
\maketitle
\tableofcontents
\newpage

\noindent
\textbf{Status:} Binding Governance Index \\
\textbf{Authority:} Adopted by the Governing Body pursuant to the Consolidated Governance Document \\
\textbf{Purpose:} Enumerate the exclusive set of Governance Schedules in force \\
\textbf{Precedence:} Subordinate to the Consolidated Governance Document; controlling as to schedule existence \\
\textbf{Mutability:} Revisable only by Extraordinary Process

\section{Purpose}

This Schedule enumerates the complete and exclusive set of Governance Schedules adopted under the Consolidated Governance Document.

No schedule has force or effect unless listed in this Index.

\section{Adopted Governance Schedules}

\begin{itemize}
\item \textbf{Schedule A (10.10): Material Action Thresholds} \\
Defines numeric and categorical thresholds that trigger Material Action treatment.

\item \textbf{Schedule B (10.20): Emergency Scope Registry} \\
Defines the exclusive persons, animals, assets, spaces, and systems within which emergency authority may be exercised.

\item \textbf{Schedule C (10.30): Delegation and Approval Matrix} \\
Defines who may approve which actions, up to what limits, and under what constraints.

\item \textbf{Schedule D (10.40): Logging and Audit Integrity Policy} \\
Defines binding requirements for logging, auditability, tamper-evidence, retention, and correction of governed records.
\end{itemize}

\section{Adoption and Removal}

The addition or removal of any Governance Schedule requires:

\begin{itemize}
\item adoption by recorded vote of the Governing Body,
\item amendment of this Schedule Index,
\item compliance with the applicable mutability requirements.
\end{itemize}

\section{Anti-Expansion Rule}

Silence, reference, or implication does not create a Governance Schedule.

Only schedules explicitly listed in this Index have force or effect.

\section{Interpretive Rule}

This Schedule must be interpreted narrowly. Where ambiguity exists, interpretation must favor:

\begin{itemize}
\item non-expansion of authority,
\item preservation of mission lock,
\item escalation to the Consolidated Governance Document.
\end{itemize}

\end{document}
