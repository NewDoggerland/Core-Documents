
\documentclass[11pt]{article}
\usepackage{ndstyle}
\setcounter{secnumdepth}{3}
\setcounter{tocdepth}{3}

\title{New Doggerland - Governance Companion Document v0.5}
\author{}
\date{}

\begin{document}
\maketitle
\tableofcontents
\newpage

\noindent
\textbf{Status:} Non-binding explanatory companion \\
\textbf{Authority:} Subordinate to ``New Doggerland - Consolidated Governance Document v0.5'' \\
\textbf{Purpose:} Explain intentions and mechanisms without creating new obligations

\section{What this document is and is not}

\subsection{What this document is}

This document is a readable explanation of how the Consolidated Governance Document is designed to function in practice. It is written to help:
\begin{itemize}
\item board members,
\item staff,
\item partner organizations,
\item donors,
\item auditors, and
\item future stewards
\end{itemize}
understand the system quickly and accurately.

\subsection{What this document is not}

This document is not:
\begin{itemize}
\item a source of authority,
\item policy,
\item an operational manual,
\item a substitute for the Consolidated Governance Document,
\item or a mechanism to reinterpret rules.
\end{itemize}

If there is any difference between this companion and the Consolidated Governance Document, the Consolidated Governance Document controls.

\section{How to use this companion}

This companion is a \textbf{reader’s guide}. It explains \textbf{intent}, \textbf{mechanism}, and \textbf{the mental model} of the governance system. It is not a source of new obligations.

\subsection{If you only have 5 minutes}

Read, in order:
\begin{enumerate}
\item \textbf{Design intent in one paragraph}
\item \textbf{The core logic of the system}
\item \textbf{What to do when confused} (below)
\end{enumerate}

\subsection{Reader pathways}

Choose the path that matches why you are here:

\begin{itemize}
\item \textbf{Staff / contributors:} Read for boundaries and escalation routes. Focus on constraints, discretion limits, and how to route uncertainty.
\item \textbf{Stewards / governance roles:} Read for decision-shaping principles, interpretation rules, and what makes a decision legitimate.
\item \textbf{Partners / funders / external reviewers:} Read for capture-resistance, mission lock, and how conflicts are contained and made reviewable.
\end{itemize}

\subsection{The system in 60 seconds}

New Doggerland governance works like this:

\begin{enumerate}
\item \textbf{Mission lock first:} the organization has a constrained purpose-space; some actions are simply not allowed.
\item \textbf{Welfare and dignity constraints:} dog welfare and human dignity bound what can be justified, even under pressure.
\item \textbf{Lowest competent level:} decisions should be made as close to the work as competence and risk allow.
\item \textbf{Bounded discretion:} discretion exists, but must be explainable and routed through reviewable pathways.
\item \textbf{Recorded pathways:} when stakes are material, decision logic must be legible after the fact.
\end{enumerate}

\subsection{What to do when confused}

When a clause feels unclear, do \textbf{exactly this}, in order:

\begin{enumerate}
\item Identify whether the question is about \textbf{purpose}, \textbf{authority}, \textbf{process}, or \textbf{interpretation}.
\item Prefer the \textbf{most constraint-preserving} interpretation (the one least likely to expand power or undermine welfare or dignity).
\item Route the uncertainty through the \textbf{smallest} legitimate escalation path, and record the rationale if it affects procedure.
\end{enumerate}

\noindent
If you need a single test: \textbf{Could a good-faith reviewer understand why this choice was made, and see that it remained inside the mission lock?}

\section{Design intent in one paragraph}

New Doggerland’s governance is designed to resist capture, prevent private extraction, and preserve mission integrity over multiple generations. The system assumes that failure modes will occur—good-faith mistakes, ambiguous circumstances, missing information, emergencies, and occasional bad-faith actors—and it therefore encodes safeguards that force decisions into transparent, reviewable pathways with explicit limits on authority.

For the intended end state and staged pathway, see Founder Vision. It confers no authority.

\section{The core logic of the system}

\subsection{Mission lock as the first constraint}

The governance framework begins by defining the organization’s allowed purpose-space. It does this by establishing that:
\begin{itemize}
\item the organization exists solely for its stated charitable, welfare, and stewardship purposes, and
\item no activity may undermine dog welfare, human dignity, or long-term stewardship.
\end{itemize}

\subsection{Mutability and permanence}

Every section of the Consolidated Governance Document declares a mutability class. Where multiple mutability declarations apply, the most specific declaration governs. This means that subsection-level mutability overrides section-level mutability.

Unchangeable provisions are intended to remain fixed. Revisable provisions may change only through the specified governance process. Adaptive provisions may evolve within explicitly defined bounds.

\subsection{Early-stage operability}

The governance system explicitly allows for a bounded Foundational Phase during which certain governance mechanisms or operational systems may not yet be fully instantiated. This flexibility is:
\begin{itemize}
\item temporary,
\item explicitly recorded,
\item non-precedential, and
\item subject to automatic expiration.
\end{itemize}

This mechanism exists to permit safe early operation without weakening mission lock or authority constraints.

\subsection{Hierarchy determines authority, not interpretation}

The hierarchy section establishes that different document layers exist, but clarifies a critical point:
\begin{itemize}
\item \textbf{Hierarchy governs authority.}
\item \textbf{Interpretation still requires good faith and narrow reading.}
\end{itemize}

\subsection{``Material Action'' is the system’s primary trigger concept}

A large portion of the document exists to ensure that important actions cannot be performed casually.

\subsection{Recorded Decision Pathways are the audit spine}

A Recorded Decision Pathway is the minimum complete record showing:
\begin{itemize}
\item what was decided,
\item who authorized it,
\item when it happened,
\item what it affects, and
\item what authority was relied upon.
\end{itemize}

\subsection{Interpretation notes exist, but are deliberately defanged}

Interpretation notes are permitted at the lowest competent level only for non-material actions, and only where no conflict exists.

\section{Roles and separation of power}

\subsection{The Founder role is real but non-controlling}

The Founder is defined as a continuity and coherence role. The design deliberately forbids the Founder from having voting power, veto power, unilateral authority, ownership interest, or a governance seat by default.

\section{Residential continuity protections}

The residence provisions are designed as continuity and dignity protections, not as transferable benefits.

They include:
\begin{itemize}
\item non-owning residency,
\item non-transferability,
\item explicit protection against displacement,
\item termination upon abuse, and
\item constraints to preserve safety and operations.
\end{itemize}

Residency protections exist for the Founder, surviving spouse, and certain narrowly defined service-based roles. Where specified in the Consolidated Governance Document, these protections are lifetime, dignity-preserving, and non-extractive, and never create ownership, inheritance, income rights, or governance authority.

\section{Companion document maintenance rule}

This companion document may be updated for clarity and readability, but must remain strictly derived from the Consolidated Governance Document. Any update that could be interpreted as creating new obligations must be rejected.

\end{document}
