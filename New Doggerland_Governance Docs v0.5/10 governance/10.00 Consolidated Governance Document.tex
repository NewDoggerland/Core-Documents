\documentclass[11pt]{article}
\usepackage{ndstyle}
\setcounter{secnumdepth}{3}
\setcounter{tocdepth}{3}


\title{New Doggerland - Consolidated Governance Document v0.5}
\author{}
\date{}

\begin{document}
\maketitle
\tableofcontents
\newpage


\section{Purpose and Mission Lock}

New Doggerland exists solely to advance its stated charitable, welfare, and stewardship purposes. Assets, authority, and operations must not deviate from these purposes. No activity may undermine dog welfare, human dignity, or the long-term stewardship obligations of the organization. 
No activity may be treated as undermining welfare, dignity, or stewardship solely because it imposes incidental, time-limited, or mitigated burdens that are necessary to prevent greater harm or to carry out mission-aligned duties under this document, provided the necessity basis is documented contemporaneously in the applicable log, interpretation note, or Recorded Decision Pathway.

\subsection{Domain-Specific Optimization Constraint}
\label{precept:domain-specific-optimization}

New Doggerland must designate a primary beneficiary for every operational, spatial, or experiential domain.

Within a domain, New Doggerland may optimize conditions only for the designated primary beneficiary.

Within a domain, New Doggerland must respect all non-primary beneficiaries.

New Doggerland must not optimize conditions for more than one beneficiary within the same domain.

New Doggerland must not use indulgence as a design, operational, or experiential objective in any domain.

Mutability: Unchangeable


\section{Authority, Hierarchy, and Precedence}
\label{sec:authority-hierarchy}

The governing hierarchy is fixed as follows:

\begin{enumerate}
\item Legal and Regulatory Compliance Instruments
\item This Consolidated Governance Document
\item Operational Manuals
\item Policies and Procedures
\end{enumerate}

\noindent In any conflict, higher-order documents control. Legal and regulatory requirements control in all cases, regardless of any internal provision.

\noindent Subordination within this hierarchy governs authority, not interpretation. Where no direct conflict exists, good-faith interpretation at the lowest competent level is permitted only for actions that are not Material Actions, provided such interpretation does not create new authority or obligations and is recorded as an interpretation note when it affects procedure.

Interpretation notes are non-binding and do not create precedent, authority, or obligations. Interpretation notes must not be cited as independent authority in any Recorded Decision Pathway and must not be treated as policy unless formally adopted through the applicable governance or policy process.

\noindent Ambiguity alone does not constitute conflict. Only direct inconsistency between provisions triggers hierarchical override.

Mutability: Unchangeable

\section{Derivation and Compression Principle}
\label{sec:derivation-compression}

All authority within New Doggerland originates exclusively from the governing hierarchy defined in Section~\ref{sec:authority-hierarchy}.

All materials that are not part of that hierarchy, including but not limited to companion documents, public descriptions, staff handbooks, training materials, summaries, slogans, heuristics, cultural artifacts, and informal guidance, are subordinate and may exist only as derived representations of higher-order documents.

Derived materials may summarize, simplify, restate, or compress existing rules and obligations, but must not originate new authority, create new obligations, grant permissions, change thresholds, modify safeguards, or override higher-order text.

Derived materials must not be relied upon to determine rights, duties, permissions, prohibitions, or enforcement outcomes.

If any conflict, ambiguity, or inconsistency exists between a higher-order document and a derived material, the higher-order document governs without exception.

No cultural, informal, or symbolic expression has independent authority or interpretive weight.

External assurance documents (including donor, regulator, or auditor summaries) are derived representations and confer no authority.

Mutability: Unchangeable


\section{Definitions}

\subsection{Founder}
\label{def:founder}

The Founder is a defined organizational role responsible for originating and stewarding the foundational structure, mission articulation, and long-horizon coherence of New Doggerland.

The initial Founder is the natural person who established New Doggerland and is identified as such in the organization’s formation records.

The Founder role:
\begin{itemize}
\item is singular and may be occupied by only one person at a time
\item does not constitute an office, officer position, ownership interest, or governance seat
\item confers no voting power, veto power, or unilateral authority
\item holds the non-voting Governing Body Chair role by default unless declined in writing
\item may be vacated, retired from, or reassigned only in accordance with this document
\end{itemize}

Upon retirement or voluntary transition from active stewardship, the incumbent Founder enters the Founder Steward role as defined in Section~\ref{subsec:founder-steward}.

Upon incapacity, entry into the Founder Steward role is permitted only if the Governing Body records a finding that the incumbent Founder can reliably perform the Chair and advisory functions described in Section~\ref{subsec:founder-steward}.

Such entry:
\begin{itemize}
\item does not terminate the Founder’s identity or historical status
\item represents a retreat from day-to-day operational and physical work
\item preserves continuity of mission articulation and long-horizon coherence
\end{itemize}

Mutability: Unchangeable

\subsection{Stewardship}
\label{def:stewardship}

Stewardship is an independent oversight function responsible solely for continuity, integrity, and failure backstopping where governance mechanisms cannot operate as designed.

Stewardship:
\begin{itemize}
\item is not a Governing Body
\item holds no policy-making, voting, or directive authority
\item exists only to preserve procedural continuity and prevent deadlock or capture
\item must not be occupied by any person currently serving on the Governing Body
\item must not be occupied by the Founder, Founder Steward, or any executive staff
\end{itemize}

Stewardship authority is limited strictly to the functions explicitly enumerated in this document.

Mutability: Unchangeable

\subsection{Steward’s Office}
\label{def:stewards-office}

The Steward’s Office is the minimal administrative expression of Stewardship required to execute continuity functions explicitly granted by this document.

The Steward’s Office:
\begin{itemize}
\item may act only where this document explicitly authorizes Stewardship intervention
\item must act mechanically and without discretion
\item must document all actions taken
\item must dissolve automatically once the triggering condition is resolved
\end{itemize}

The Steward’s Office must not:
\begin{itemize}
\item issue policy
\item influence governance decisions
\item override votes
\item persist beyond its authorized scope
\end{itemize}
For avoidance of doubt, the requirement that the Steward’s Office act mechanically and without discretion prohibits independent judgment, prioritization, or policy interpretation. This does not prohibit the minimal application of Bounded Discretion solely where explicitly required to execute a continuity function already authorized by this document, provided that such discretion is fully documented and remains strictly constrained to execution mechanics.


Mutability: Unchangeable

\subsection{Supermajority}
\label{def:supermajority}

A Supermajority is an affirmative vote of not less than two-thirds (2/3) of all active voting seats of the Governing Body, rounded up to the next whole number where necessary.

Abstentions, absences, recusals, and vacancies do not reduce the number of votes required unless this document explicitly states otherwise.

Mutability: Unchangeable

\subsection{Extraordinary Process}
\label{def:extraordinary-process}

An Extraordinary Process is a modification process that requires all of the following:

\begin{itemize}
\item a Supermajority vote of the Governing Body;
\item advance written notice of the proposed modification, delivered to the Governing Body Chair and all active voting members;
\item explicit identification of the clauses affected and their current mutability classification;
\item a recorded justification stating why use of an Extraordinary Process is required to protect mission, safety, dignity, or stewardship constraints under this document.
\end{itemize}

An Extraordinary Process must not be completed through emergency authority, delegation, administrative action, or procedural shortcut.

Mutability: Unchangeable


\subsection{Active Voting Seats}
\label{def:active-voting-seats}

Active Voting Seats are Governing Body seats that are currently filled by members whose terms are valid and not suspended or disqualified due to removal, resignation, Credible Impairment determination, classification-related suspension, or expiration of term.

Vacant seats, expired terms, and seats subject to formal suspension or disqualification do not constitute Active Voting Seats for purposes of quorum, Supermajority calculations, or voting thresholds unless this document explicitly states otherwise.

Mutability: Unchangeable


\subsection{Properly Requested Meeting}
\label{def:properly-requested-meeting}

A Properly Requested Meeting is a Governing Body meeting requested in accordance with all of the following conditions:

\begin{itemize}
\item the request is made in writing
\item the request specifies the matters to be addressed
\item the request is delivered to the Governing Body Chair and all voting members
\item the request is supported by not less than one-third (1/3) of all active voting members
\item the requested meeting date is not less than seven (7) days and not more than thirty (30) days from the date of request
\end{itemize}

Failure by the Governing Body Chair to convene a Properly Requested Meeting within the requested window constitutes refusal for purposes of any procedural backstop in this document.

Mutability: Unchangeable

\subsection{Materially Affects}
\label{def:materially-affects}

A decision materially affects a person, asset, domain, or obligation if it meets one or more of the following conditions:

\begin{itemize}
\item creates, modifies, or terminates a right, duty, protection, or restriction
\item alters safety conditions, welfare outcomes, or dignity protections
\item commits, transfers, encumbers, or risks organizational assets or land
\item establishes precedent that would reasonably influence future decisions of a similar type
\item changes governance structure, authority allocation, or review thresholds
\end{itemize}

A decision must be treated as materially affecting if reasonable disagreement exists as to its impact, provided that the disagreement is grounded in plausible, non-speculative effects within a reasonable operational time horizon based on facts known or reasonably available at the time. Abstract, hypothetical, or purely downstream disagreement does not, by itself, render an action materially affecting.


Mutability: Unchangeable

\subsection{Material Action}
\label{def:material-action}

A Material Action is any action, decision, omission, authorization, or override that materially affects (as defined in Section~\ref{def:materially-affects}) any person, asset, domain, obligation, safety condition, welfare outcome, dignity protection, or governance mechanism.

A Material Action includes:
\begin{itemize}
\item any decision that requires a recorded vote under this document
\item any invocation of emergency authority
\item any override of a policy, procedure, safeguard, or default restriction
\item any action that establishes precedent that would reasonably influence future decisions of a similar type
\item routine actions that are reversible, role-scoped, and logged do not become Material Actions solely due to hypothetical precedent concerns or abstract downstream disagreement.
\end{itemize}

If reasonable disagreement exists, as qualified under Section~\ref{def:materially-affects}, as to whether an action is material, it must be treated as a Material Action.


Mutability: Unchangeable

\subsection{Recorded Decision Pathway}
\label{def:recorded-decision-pathway}

A Recorded Decision Pathway is the minimum complete written record that shows how and why a Material Action (as defined in Section~\ref{def:material-action}) was authorized and executed.

A Recorded Decision Pathway must include all of the following:
\begin{itemize}
\item a unique identifier for the action record
\item the actor or body that authorized the action
\item the date and time of authorization
\item the triggering conditions and the specific decision being made
\item the scope of impact, including what is affected and what is not affected
\item the authority basis, including the specific section(s) of this document or operational manual procedure relied upon
\item any required vote record, quorum status, recusals, and abstentions where applicable
\item the logging location and the person responsible for completing any follow-on documentation
\end{itemize}

If any required element is missing, the Recorded Decision Pathway is invalid.

For emergency actions, a Recorded Decision Pathway exists upon creation of a compliant emergency log stub containing, at minimum:
\begin{itemize}
\item the unique identifier
\item the actor invoking emergency authority
\item the triggering conditions
\item the scope and emergency procedure invoked
\end{itemize}

All missing elements must be completed within the time limits required by the applicable operational manual emergency procedure.

Mutability: Unchangeable

\subsection{Bounded Discretion}
\label{def:bounded-discretion}

Bounded Discretion is the minimum unavoidable judgment exercised to apply a rule whose trigger or scope cannot be executed mechanically.

Bounded Discretion is permitted only if:
\begin{itemize}
\item the triggering standard cannot be applied using objective criteria alone,
\item the actor documents the basis for the judgment contemporaneously,
\item the judgment is limited to the narrowest scope consistent with preventing harm or meeting the duty, and
\item the action remains fully subject to logging, review, and classification requirements under this document.
\end{itemize}

Bounded Discretion must not be used to:
\begin{itemize}
\item expand authority beyond what is explicitly granted,
\item bypass required votes, recusals, or quorum safeguards,
\item create precedent beyond the specific facts documented in the record.
\end{itemize}

Mutability: Unchangeable

\subsection{Routine, Reversible, and Role-Scoped}
\label{def:rrr}

For purposes of this document and the Operational Manuals:

\begin{itemize}
\item \textbf{Routine} means an action that is customary for the role, low-complexity, and does not create or modify rights, restrictions, or governance allocations.
\item \textbf{Reversible} means the action can be undone or returned to the prior baseline without requiring governance approval, without material cost, and without creating ongoing obligations or irreversible reliance.
\item \textbf{Role-Scoped} means the action falls within a written role description, role charter, or operational delegation record that is maintained and reviewable.
\end{itemize}

Mutability: Unchangeable


\subsection{External Governance Arbiter}
\label{def:external-governance-arbiter}

An External Governance Arbiter is an independent individual or body retained solely to provide neutral adjudication, review, or approval functions expressly required by this document.

An External Governance Arbiter must satisfy all of the following conditions:
\begin{itemize}
\item is not a current or former member of the Governing Body
\item is not the Founder, Founder Steward, or any executive or employee of the organization
\item has no material financial, familial, or supervisory relationship with any governed party
\item has not provided paid services to the organization within the preceding twenty-four (24) months, except for prior arbiter service
\item has no contingent compensation tied to outcomes
\item is selected through a documented process designed to preserve independence
\end{itemize}

An External Governance Arbiter:
\begin{itemize}
\item holds no governing authority
\item may not issue policy
\item may act only within the scope explicitly granted by this document
\item must issue written findings when required
\end{itemize}

Failure to meet any eligibility condition renders the Arbiter appointment invalid.

Mutability: Unchangeable

\subsection{Controlled Subsidiary}
\label{def:controlled-subsidiary}

A Controlled Subsidiary is any legal entity in which New Doggerland directly or indirectly:
\begin{itemize}
\item owns more than fifty percent (50\%) of equity, membership interests, or voting control rights, or
\item has contractual power to appoint a majority of governing managers, directors, or equivalent controlling decision-makers, or
\item has effective control over budget approval, profit distribution decisions, or liquidation decisions.
\end{itemize}

For avoidance of doubt, a for-profit entity may be a Controlled Subsidiary.

Mutability: Unchangeable

\subsection{Subsidiary Governance Charter}
\label{def:subsidiary-governance-charter}

A Subsidiary Governance Charter is a written instrument adopted by the Governing Body that defines the subsidiary’s:
\begin{itemize}
\item purpose and permitted activity scope,
\item governance structure and appointment method,
\item oversight and reporting obligations to New Doggerland,
\item conflict-of-interest and related-party transaction rules,
\item profit distribution and reserve policy,
\item shutdown, divestment, or dissolution triggers and process.
\end{itemize}

A Subsidiary Governance Charter is subordinate to this Consolidated Governance Document pursuant to Section~\ref{sec:authority-hierarchy}.

Mutability: Unchangeable

\subsection{Related-Party Transaction}
\label{def:related-party-transaction}

A Related-Party Transaction is any transaction, contract, lease, service agreement, loan, transfer, or in-kind arrangement between:
\begin{itemize}
\item New Doggerland and a Controlled Subsidiary,
\item two or more Controlled Subsidiaries,
\item New Doggerland or a Controlled Subsidiary and the Founder, Founder Steward, any Governing Body member, any executive staff, or any member of their households,
\item New Doggerland or a Controlled Subsidiary and any entity in which any of the above persons holds a Material Financial Interest (as defined in Section~\ref{def:material-financial-interest}).
\end{itemize}

Mutability: Unchangeable



\subsection{Independent Supervisor Layer}
\label{def:independent-supervisor-layer}

An Independent Supervisor Layer is a bona fide supervisory role that satisfies all of the following conditions:

\begin{itemize}
\item the supervisor is not a member of the Governing Body
\item the supervisor is not the Founder or Founder Steward
\item the supervisor does not report to the supervised person
\item the supervisor has documented authority to evaluate performance, enforce standards, and recommend discipline or termination
\item the supervisor’s compensation, role security, and evaluation are not controlled by the supervised person
\end{itemize}

A nominal, ceremonial, or conflicted supervisory arrangement does not constitute an Independent Supervisor Layer.

Mutability: Unchangeable

\subsection{Credible Impairment}
\label{def:credible-impairment}

Credible Impairment is a demonstrable condition that materially prevents a person from reliably performing the essential duties of their role.

A finding of Credible Impairment requires documented evidence of one or more of the following:

\begin{itemize}
\item persistent inability to participate meaningfully in required duties
\item sustained cognitive, physical, or psychological limitation affecting judgment, reliability, or capacity
\item medically documented incapacity relevant to role performance
\item repeated failure to meet role obligations despite reasonable accommodation
\end{itemize}

Credible Impairment must not be based solely on:
\begin{itemize}
\item disagreement, dissent, or unpopular positions
\item age, disability status, or protected characteristics
\item temporary illness, stress, or isolated incidents
\item uncorroborated allegations or subjective impressions
\end{itemize}

Any determination of Credible Impairment must be documented, reviewable, and subject to the same procedural safeguards applicable to removal for cause.

Mutability: Unchangeable

\subsection{Material Financial Interest}
\label{def:material-financial-interest}

A Material Financial Interest exists when a person has any direct or indirect financial stake that could reasonably influence, or be perceived to influence, their judgment, independence, or conduct with respect to the organization.

A Material Financial Interest includes, but is not limited to:
\begin{itemize}
\item ownership, equity, profit-sharing, or revenue participation in a vendor, contractor, partner, or counterparty
\item receipt of compensation, consulting fees, retainers, commissions, or success-based payments
\item outstanding loans, guarantees, or financial obligations involving the organization
\item expectation of future compensation or benefit contingent on organizational action
\item financial interests held by a spouse, domestic partner, or household member that would reasonably affect independence
\end{itemize}

A financial interest is material regardless of:
\begin{itemize}
\item size or percentage if the interest is non-trivial to the holder
\item whether compensation is labeled informal, deferred, conditional, or non-cash
\item whether the interest is routed through an intermediary entity
\end{itemize}

A financial interest is not material solely because of:
\begin{itemize}
\item compensation received as an employee under standard employment terms
\item reimbursement for documented expenses
\item de minimis gifts disclosed and approved under policy
\end{itemize}

Any reasonable doubt as to materiality must be resolved in favor of classification as material.

Mutability: Unchangeable

\subsection{Independently Valued Housing Benefit}
\label{def:independently-valued-housing}

An Independently Valued Housing Benefit is a non-transferable, non-owning residential use right provided solely for organizational continuity, security, or operational necessity.

Such a benefit:
\begin{itemize}
\item must be supported by a documented business-necessity rationale
\item must be independently valued at fair market rental value by a qualified, disinterested assessor
\item must be reviewed periodically as required by this document
\item must be disclosed internally and in donor-facing transparency materials
\item must be evaluated and reported consistently with applicable tax, disclosure, and reporting requirements, including inclusion in total compensation analysis where required
\end{itemize}
Failure to meet these requirements converts the benefit into a Material Financial Interest.

Mutability: Unchangeable

\subsection{Independently Verifiable Compensation Data}
\label{def:independently-verifiable-compensation-data}

Independently Verifiable Compensation Data is compensation information sourced from entities or datasets that are external to the organization, methodologically transparent, and reasonably resistant to manipulation.

Independently Verifiable Compensation Data must satisfy all of the following:
\begin{itemize}
\item originate from a source not controlled by the organization or the compensated individual
\item disclose methodology or sampling basis sufficient to assess reliability
\item be current within a reasonable time horizon relative to the compensation decision
\item be accessible for review, audit, or replication
\end{itemize}

Acceptable sources include, but are not limited to:
\begin{itemize}
\item nationally or regionally recognized compensation surveys
\item government labor statistics or public labor databases
\item reputable industry compensation benchmarks
\item independently conducted third-party studies
\end{itemize}

Unacceptable sources include:
\begin{itemize}
\item internally generated surveys or estimates
\item anecdotal reports or recruiter representations without supporting data
\item non-transparent web aggregators lacking methodological disclosure
\item sources with a direct financial interest in the compensation outcome
\end{itemize}

Where multiple credible sources diverge, the Governing Body must document the rationale for selecting the benchmark used.

Any reasonable doubt as to independence or verifiability must be resolved against use of the data.

Mutability: Unchangeable

\subsection{Upper End of Prevailing Market Norms}
\label{def:upper-end-market}

The Upper End of Prevailing Market Norms means compensation at or above the seventy-fifth (75th) percentile of independently verifiable market compensation for comparable roles, responsibilities, experience, and geographic context.

Determination must:
\begin{itemize}
\item use Independently Verifiable Compensation Data (as defined in Section~\ref{def:independently-verifiable-compensation-data})
\item rely on multiple reputable sources where available
\item default to the higher benchmark where credible sources diverge
\end{itemize}

If percentile data is unavailable for a role, the Governing Body must document a conservative proxy methodology designed to approximate the seventy-fifth (75th) percentile and must disclose the limitation.

Any reasonable doubt as to percentile placement must be resolved upward.

Mutability: Unchangeable

\subsection{Compensation Philosophy Framework}
\label{def:compensation-philosophy}

A Compensation Philosophy Framework is a documented, role-tiered approach to compensation that establishes target ranges, benchmarking methods, and adjustment criteria consistent with organizational sustainability, dignity protections, and non-extractive principles.

The framework must:
\begin{itemize}
\item define role categories with comparable responsibility and risk profiles
\item specify target compensation ranges for each category using independent benchmarks
\item identify conditions under which compensation may be set above or below target
\item preserve the Dignity Floor under all circumstances
\item prohibit retaliation, coercion, or mission-based underpayment
\item be documented and reviewable
\end{itemize}

Failure to maintain or follow an approved Compensation Philosophy Framework constitutes a governance failure requiring remediation.

Mutability: Unchangeable


\subsection{Dignity Floor}
\label{def:dignity-floor}

The Dignity Floor is the minimum compensation level below which no participating individual may be reduced under any hardship, emergency, or continuity mechanism.

The Dignity Floor must be sufficient to meet basic living needs for the applicable role holder and must be determined using objective, externally verifiable criteria.

At a minimum, the Dignity Floor must account for:
\begin{itemize}
\item housing costs at modest, non-luxury levels within the relevant geographic area
\item food sufficient for nutritional adequacy
\item basic utilities and communications
\item healthcare access or insurance continuity
\item transportation necessary to perform required duties
\end{itemize}

The Dignity Floor must be calculated using:
\begin{itemize}
\item publicly available cost-of-living data
\item government poverty, subsistence, or living-wage benchmarks
\item regional housing and expense indices
\end{itemize}

The Dignity Floor must not:
\begin{itemize}
\item rely on individual frugality, sacrifice, or mission alignment
\item assume unpaid labor, deferred compensation, or future make-whole promises
\item be set below any legally mandated minimums
\end{itemize}

Where multiple benchmarks exist, the highest applicable benchmark must be used unless the Governing Body documents a clear, non-extractive justification for an alternative.

The Dignity Floor determination must be documented, reviewable, and disclosed to affected persons prior to any compensation reduction.

Any reasonable doubt as to adequacy must be resolved in favor of the individual.

Mutability: Unchangeable

\section{Mission Statement}
\label{sec:mission-statement}

\subsection{Status and Authority}
\label{subsec:mission-status}

This Mission Statement is non-binding. It exists to clarify intent and orientation. It does not grant authority, create obligations, or supersede any provision of this Consolidated Governance Document or any governing document derived from it. In the event of any conflict, this Mission Statement is subordinate to this Consolidated Governance Document and to the governing hierarchy stated in Section \ref{sec:authority-hierarchy}.

Mutability: Revisable

\subsection{Mission}
\label{subsec:mission-text}

New Doggerland exists to restore and protect the dignity of dogs and people through long-term stewardship, direct care, and community access to safe, diverse environments for coexistence, where such access is consistent with animal welfare and dignity constraints. All New Doggerland operations exist to support and sustain animal rescue, sanctuary, and adoption, and may not operate independently of that purpose. The organization proceeds with restraint, prioritizing animal and human dignity over growth, efficiency, popularity, or scale, and accepting limits on expansion, cost, or scope when required to uphold that standard. New Doggerland advances incrementally, adding features and services only as ethical capacity and funding allow, and refusing activities or funding that would compromise its mission or shift its intentions. It is structured to endure rather than to scale, to remain accountable to the future, and to continue providing care and service even under constraint.

Mutability: Unchangeable


\section{Governance Bodies}  


\subsection{Governing Authority}
\label{subsec:governing-authority}

The Governing Body holds ultimate authority over organizational decisions, subject to the constraints of this document and applicable law. No individual may exercise unilateral control over governance outcomes. 

Mutability: Revisable

\subsection{Controlled Subsidiary Creation Authority}
\label{subsec:subsidiary-creation-authority}

Creation, acquisition, or assumption of control of any Controlled Subsidiary (as defined in Section~\ref{def:controlled-subsidiary}) constitutes a Material Action under Section~\ref{def:material-action} and requires a Recorded Decision Pathway under Section~\ref{def:recorded-decision-pathway}.

No Controlled Subsidiary may be created, acquired, or controlled unless:
\begin{itemize}
\item the Governing Body approves the action by Supermajority vote as defined in Section~\ref{def:supermajority},
\item a Subsidiary Governance Charter (as defined in Section~\ref{def:subsidiary-governance-charter}) is adopted prior to execution of any controlling instrument, and
\item the Charter explicitly states the subsidiary’s permitted scope and the prohibition on the subsidiary creating new authority for New Doggerland beyond this document.
\end{itemize}

Mutability: Unchangeable

\subsection{Subsidiary Oversight and Operational Independence}
\label{subsec:subsidiary-oversight-independence}

The Governing Body retains ultimate oversight responsibility for all Controlled Subsidiaries. Such oversight is limited to:
\begin{itemize}
\item approval of the Subsidiary Governance Charter and any amendments,
\item approval of annual subsidiary budgets and capital commitments above Charter-defined thresholds,
\item appointment or removal of subsidiary managers or directors only as permitted by the Charter,
\item review of quarterly reports required by Section~\ref{subsec:subsidiary-profit-policy} and Section~\ref{subsec:subsidiary-mission-protection},
\item enforcement actions and shutdown/divestment actions under Section~\ref{subsec:subsidiary-shutdown-divestment}.
\end{itemize}

Except for the oversight functions explicitly enumerated in this subsection and in the applicable Subsidiary Governance Charter, the Governing Body must not direct day-to-day subsidiary operations.

No subsidiary governance instrument, policy, or practice may be treated as creating authority over New Doggerland beyond the governing hierarchy in Section~\ref{sec:authority-hierarchy}.

Mutability: Unchangeable

\subsection{Subsidiary Conflicts and Related-Party Transactions}
\label{subsec:subsidiary-coi}

All Related-Party Transactions (as defined in Section~\ref{def:related-party-transaction}) constitute Material Actions under Section~\ref{def:material-action} and require a Recorded Decision Pathway under Section~\ref{def:recorded-decision-pathway}.

No Related-Party Transaction may be approved unless:
\begin{itemize}
\item all persons with a Material Financial Interest (as defined in Section~\ref{def:material-financial-interest}) are fully disclosed and fully recused from discussion, influence, and vote,
\item the transaction is documented as fair, non-extractive, and at or better than market terms to New Doggerland or the affected subsidiary,
\item the Governing Body records the basis for fairness contemporaneously in the Recorded Decision Pathway.
\end{itemize}

A Subsidiary Governance Charter must include a mechanical conflict screening workflow and must prohibit informal or undocumented arrangements between New Doggerland and any Controlled Subsidiary.

Mutability: Unchangeable

\subsection{Subsidiary Profit Distribution and Reserve Policy}
\label{subsec:subsidiary-profit-policy}

Each Controlled Subsidiary must operate under a documented profit distribution and reserve policy contained in its Subsidiary Governance Charter.

The default permitted direction of value transfer is from subsidiary to New Doggerland. Transfers from New Doggerland to any Controlled Subsidiary are prohibited unless:
\begin{itemize}
\item structured as a written, market-rate loan or documented capital contribution,
\item approved as a Material Action by Supermajority vote under Section~\ref{def:supermajority},
\item documented with a Recorded Decision Pathway under Section~\ref{def:recorded-decision-pathway},
\item and demonstrated to be necessary to preserve mission-aligned continuity without creating dependency or extraction risk.
\end{itemize}

Subsidiary distributions to New Doggerland must not be represented as donor-directed unless legally restricted and recorded as such. Subsidiary profits must not be relied upon to justify reduction of any Dignity Floor obligations under Section~\ref{def:dignity-floor} or to subsidize extractive labor practices.

Mutability: Unchangeable

\subsection{Subsidiary Shutdown, Divestment, or Dissolution Authority}
\label{subsec:subsidiary-shutdown-divestment}

Shutdown, divestment, dissolution, or loss of control of any Controlled Subsidiary constitutes a Material Action under Section~\ref{def:material-action} and requires a Recorded Decision Pathway under Section~\ref{def:recorded-decision-pathway}.

The Governing Body may authorize shutdown or divestment by Supermajority vote under Section~\ref{def:supermajority} if any of the following conditions are met:
\begin{itemize}
\item the subsidiary materially undermines the Mission Lock in Section~1,
\item the subsidiary creates material welfare, dignity, safety, or stewardship risk as defined in Section~\ref{def:materially-affects},
\item the subsidiary requires ongoing subsidy from New Doggerland in violation of Section~\ref{subsec:subsidiary-profit-policy},
\item the subsidiary presents credible compliance risk that cannot be mitigated through narrower controls.
\end{itemize}

A Subsidiary Governance Charter must include a mechanical shutdown and wind-down procedure that prioritizes legal compliance, staff dignity protections, and prevention of harm to animals and persons.

Mutability: Unchangeable

\subsection{Mission Protection Under Subsidiary Dominance}
\label{subsec:subsidiary-mission-protection}

If any Controlled Subsidiary becomes operationally dominant such that reasonable risk exists of mission inversion or governance capture, the Governing Body must initiate a mandatory dominance review as a Material Action under Section~\ref{def:material-action}.

For purposes of this subsection, operational dominance exists if one or more of the following conditions holds for two consecutive quarterly reporting periods:
\begin{itemize}
\item the subsidiary accounts for a majority of consolidated gross revenue,
\item the subsidiary accounts for a majority of consolidated payroll or headcount,
\item the subsidiary’s contractual obligations or liabilities would materially constrain New Doggerland’s ability to comply with the Mission Lock in Section~1.
\end{itemize}

The dominance review must:
\begin{itemize}
\item document capture risks and mission inversion risks,
\item identify controls to preserve mission integrity, including but not limited to tighter Charter constraints, governance separation, reserve rules, or divestment planning,
\item and be recorded as a Recorded Decision Pathway under Section~\ref{def:recorded-decision-pathway}.
\end{itemize}

Failure to initiate or record a required dominance review constitutes a governance process failure requiring remediation under Section~\ref{sec:classification-accountability}.

Mutability: Unchangeable

\subsection{Founder Strategic Articulation}
\label{subsec:strategic-articulation}

The Founder (as defined in Section~\ref{def:founder}) and the Founder Steward (as defined in Section~\ref{subsec:founder-steward}) may issue written Strategic Articulations describing evolving structure, priorities, or implementation pathways consistent with the mission.

Any Strategic Articulation:
\begin{itemize}
\item carries no binding authority,
\item must be formally reviewed by the Governing Body,
\item must be placed on the agenda of the next regular Governing Body meeting, and
\item must have its consideration recorded in the meeting record.
\end{itemize}

Mutability: Unchangeable


\subsection{Governing Body Composition, Selection, and Removal}

\subsubsection{Composition}
The Governing Body must consist of an odd number of voting members. The Governing Body must have no fewer than five voting members.

Mutability: Revisable

\subsubsection{Eligibility and Disqualification}
A person may serve on the Governing Body only if they can act independently and disclose conflicts.

A person may serve on the Governing Body only if they demonstrate a sustained, good-faith commitment to animal welfare, human welfare, or responsible animal stewardship, as evaluated through the organization’s formal Board vetting process.

Governing Body membership is conditional on continued compliance with this requirement. Discovery of disqualifying conduct triggers mandatory review and removal in accordance with Governing Body removal procedures.


A person must not serve on the Governing Body if they:
\begin{itemize}
\item are the Founder (as defined in Section~\ref{def:founder})
\item are the Founder’s spouse
\item are a member of the Founder’s household
\item are a current employee whose role reports directly to the Governing Body without an Independent Supervisor Layer (as defined in Section~\ref{def:independent-supervisor-layer})
\item have a Material Financial Interest (as defined in Section~\ref{def:material-financial-interest}) in vendors, partners, or contracts with the organization
\item have ever been convicted of, or are subject to a substantiated judicial or regulatory finding for, animal cruelty, animal abuse, animal neglect, or cruelty toward humans

\end{itemize}

Mutability: Unchangeable

\subsubsection{Selection and Replacement}
The Governing Body must fill vacancies by vote.

The Governing Body must not appoint a replacement without documenting:
\begin{itemize}
\item the selection method used
\item the candidate’s disclosed conflicts
\item the reason the candidate satisfies independence requirements
\end{itemize}

Mutability: Revisable

\subsubsection{Terms and Term Limits}
Governing Body members serve fixed terms of 24 months.

A Governing Body member must not serve more than two consecutive terms.

Mutability: Unchangeable

\subsubsection{Staggering Requirement}
Terms must be staggered so that no more than a simple majority of seats expire within any 90-day period.

Mutability: Unchangeable

\subsubsection{Removal}
The Governing Body may remove a member for cause by Supermajority vote as defined in Section~\ref{def:supermajority}.


Cause includes:
\begin{itemize}
\item concealment of conflicts
\item repeated nonattendance that prevents quorum
\item retaliation
\item misuse of authority
\item Credible Impairment (as defined in Section~\ref{def:credible-impairment}) preventing reliable service

\end{itemize}

The Governing Body must document the cause and the vote record for any removal.

Mutability: Unchangeable

\subsubsection{Quorum and Voting}
Quorum requires a majority of all Active Voting Seats.

A decision that materially affects (as defined in Section~\ref{def:materially-affects}) welfare, safety, dignity, assets, land, or dissolution requires a recorded vote and must not be made without quorum.


Mutability: Unchangeable

\subsubsection{Vacancy Handling}
If a vacancy causes loss of quorum, the Governing Body must prioritize restoring quorum before making irreversible decisions.

Mutability: Unchangeable

\subsection{Non-Voting Governing Body Chair (Procedural Steward Role)}
\label{subsec:governing-body-chair}

\subsubsection{Role Definition.}

The Governing Body Chair is a procedural role that presides over Governing Body meetings and safeguards procedural integrity.

The Chair role:
\begin{itemize}
\item does not constitute an office, officer position, or governance seat
\item confers no voting power, veto power, or unilateral authority
\item exists solely to facilitate orderly process and continuity

\end{itemize}

Mutability: Revisable


\subsubsection{Founder Chair Default (Non-Voting).}
The Founder (as defined in Section~\ref{def:founder}) occupies the non-voting Governing Body Chair role unless the Founder declines the position in writing, subject only to removal as permitted under Section~\ref{subsec:chair-removal}.

Mutability: Unchangeable

\subsubsection{Voting Prohibition.}
The Governing Body Chair must not vote on any Governing Body matter.

Mutability: Unchangeable

\subsubsection{Quorum Exclusion.}
The Governing Body Chair does not count toward quorum unless separately appointed as a voting member.

Mutability: Unchangeable

\subsubsection{Member Status Separation.}
The Governing Body Chair is not a voting member by virtue of holding the Chair role.

Mutability: Unchangeable

\subsubsection{Authority Limitation.}
The Governing Body Chair holds no unilateral authority over Governing Body decisions, appointments, removals, or votes.

Mutability: Unchangeable

\subsubsection{Agenda Control Safeguard.}
The Governing Body may amend the meeting agenda by majority vote at any time.

Mutability: Unchangeable

\subsubsection{Chair Neutrality and Facilitation Constraints.}

The Governing Body Chair must facilitate meetings in a procedurally neutral manner.

The Chair must:
\begin{itemize}
\item maintain a fair speaking order and reasonable time limits
\item restate motions and vote questions verbatim or by reference to written text, without persuasive framing
\item apply agenda sequencing consistently unless amended by the Governing Body
\item ensure all required votes, recusals, and quorum status are recorded
\end{itemize}

The Chair must not:
\begin{itemize}
\item summarize or characterize debate in a manner that advocates for an outcome
\item suppress agenda items properly requested under this document
\item condition recognition, access to the floor, or scheduling on agreement, deference, or silence
\end{itemize}

Any voting member may require that a Chair neutrality concern be entered into the meeting record as a procedural note.

Mutability: Unchangeable


\subsubsection{Meeting Convening Backstop.}

If the Governing Body Chair fails or refuses to convene a regular meeting or a Properly Requested Meeting (as defined in Section~\ref{def:properly-requested-meeting}), a simple majority of voting members may convene the meeting without the Chair.

Mutability: Unchangeable


\subsubsection{Tie-Breaking Explicitly Barred.}
The Governing Body Chair must not cast tie-breaking votes.

Mutability: Unchangeable

\subsubsection{Chair Removal.}
\label{subsec:chair-removal}


The Governing Body Chair may be removed only by the Governing Body as follows:

\begin{itemize}
\item Removal requires a Supermajority vote as defined in Section~\ref{def:supermajority}.
\item Removal additionally requires written confirmation by a pre-approved External Governance Arbiter (as defined in Section~\ref{def:external-governance-arbiter}) that:
\begin{itemize}
\item the process used was procedurally fair,
\item the grounds are documented and non-pretextual, and
\item the removal is not retaliation for dissent, reporting, cooperation with review, or other protected conduct under this document.
\end{itemize}
\end{itemize}

Permissible grounds for Chair removal include:
\begin{itemize}
\item repeated failure or refusal to convene required meetings, including Properly Requested Meetings (as defined in Section~\ref{def:properly-requested-meeting})
\item repeated procedural misconduct that materially impairs governance function
\item Credible Impairment (as defined in Section~\ref{def:credible-impairment}) preventing reliable performance of Chair duties
\item concealment of a Material Financial Interest (as defined in Section~\ref{def:material-financial-interest}) relevant to Chair process
\item obstruction of records, agendas, minutes, vote recording, or required logging
\end{itemize}

The Governing Body must document, at minimum:
\begin{itemize}
\item the specific grounds relied upon
\item the vote record
\item any recusals and conflicts
\item the Arbiter’s written confirmation
\end{itemize}

If the Chair is removed, the Governing Body must appoint an interim procedural Chair by majority vote from among non-disqualified persons who are not voting members, or may appoint a voting member solely to serve as interim procedural Chair without conferring additional authority beyond this Chair role.

Mutability: Unchangeable

\subsection{Founder Steward Role}
\label{subsec:founder-steward}

\subsubsection{Definition}

The Founder Steward is a post-operational stewardship state of the Founder role, analogous to a Founder Emeritus position.

The Founder Steward:
\begin{itemize}
\item represents a deliberate retreat from routine physical, managerial, or operational labor
\item preserves the Founder’s role as the long-horizon integrative mind of the organization
\item serves as cultural, architectural, and systemic continuity anchor
\item retains the non-voting Governing Body Chair role unless declined in writing
\end{itemize}

Mutability: Unchangeable

\subsubsection{Advisory Function}

The Founder Steward may issue non-binding guidance regarding architectural coherence, sequencing, and system integration.

Such guidance:
\begin{itemize}
\item carries no directive authority
\item may be directed to the Governing Body or executive staff
\item does not create obligations, expectations, or presumptions of adoption
\item remains fully subject to override or disregard by the Governing Body
\end{itemize}

Mutability: Revisable

\subsubsection{Aesthetic Stewardship and Directional Authority}


New Doggerland establishes an Aesthetic Stewardship Domain governing visual, spatial, material, atmospheric, and experiential coherence across the estate.

The Founder Steward role holds exclusive authority to issue aesthetic compatibility determinations within the Aesthetic Stewardship Domain while the role exists.
This authority attaches to the role, not the individual.

Aesthetic compatibility determinations are non-binding unless adopted by the Governing Body through the applicable governance process.

Within the Aesthetic Stewardship Domain, the Founder Steward may determine whether proposed designs, constructions, environments, or presentations are aesthetically compatible with New Doggerland.

This authority applies to architectural language and materials; landscape treatment and environmental form; signage visual grammar and illumination style; uniforms and public-facing appearance; soundscape and ambient sensory design; and the experiential tone of guest-accessible spaces.

Aesthetic Stewardship authority must not control budgets; must not direct operations; must not manage personnel; must not override adopted policy; must not suspend safety, accessibility, labor, zoning, or regulatory compliance; and must not create binding obligations outside the Aesthetic Stewardship Domain.

Aesthetic Stewardship authority may not be delegated, subdivided, or exercised by committee.
No other role may claim aesthetic authority by analogy or precedent.

If a conflict arises between Aesthetic Stewardship and safety requirements, accessibility requirements, legal or regulatory obligations, or human or animal dignity protections, the non-aesthetic requirement prevails automatically.
No justification is required.

Mutability: Revisable only by Extraordinary Process


\subsubsection{Directional Goal Stewardship}

The Founder Steward may articulate directional goals describing what New Doggerland is becoming over time.
Directional goals describe orientation, not execution.

Directional authority must not prescribe methods; must not mandate timelines; must not bind budgets; must not compel operational actions; and must not override Governing Body governance or fiduciary duty.

The Founder Steward may state that a proposal is directionally incompatible with New Doggerland.
Such a statement does not block adoption, does not invalidate the proposal, and must be recorded alongside the decision if the proposal proceeds.

Mutability: Revisable only by Extraordinary Process


\subsubsection{Advisory Continuity Protection}

After step-down or role transition, the Founder Steward retains the right to offer non-binding advisory input on aesthetic and directional matters.

No person or body may restrict access, marginalize participation, penalize expression, or retaliate against the Founder Steward for offering advisory input within scope, except where required by law, required for immediate safety, or required by a final finding of Abuse under Section~\ref{subsec:abuse}.

No successor, Governing Body, or officer is required to follow or justify rejection of advisory input.
Advisory access does not create authority.

Mutability: Revisable only by Extraordinary Process


\subsubsection{Anti-Precedent Safeguard}

No clause in this document may be interpreted to expand aesthetic or directional authority beyond what is explicitly stated.
Silence does not confer power.

No other role, committee, or office may be granted comparable discretionary authority unless explicitly created by amendment to this document.

Mutability: Unchangeable



\subsubsection{Limitations}

The Founder Steward role:

\begin{itemize}
\item Confers no ownership interest, equity interest, or financial entitlement
\item Confers no unilateral decision-making authority
\item Confers no veto power
\item Confers no operational command authority
\item May not override established governance processes
\end{itemize}

Mutability: Unchangeable


\subsubsection{Succession}

The Founder Steward role is singular.

If vacated by death or voluntary relinquishment, the role:
\begin{itemize}
\item does not automatically pass to any person
\item may be ceremonially retired
\item may be reconstituted only by explicit governance action consistent with non-control principles
\end{itemize}

Mutability: Unchangeable


\section{Ethics, Prohibitions, and Values}

\subsection{Binding Prohibitions}

No person or body may:

\begin{itemize}
\item Exploit organizational assets for private benefit
\item Compromise human dignity
\item Undermine dog welfare
\item Conceal material information
\end{itemize}

\subsection{Values and Principles}

Values guide interpretation but do not independently create enforceable obligations unless explicitly labeled as binding. Failure to consider values must be documented during panel review if relevant. 

Mutability:
\begin{itemize}
\item Prohibitions: Unchangeable
\item Values: Revisable
\end{itemize}

\subsection{Compensation and Fair Pay Constraints}

\subsubsection{Fair Compensation Framework}

All employed persons must be compensated in accordance with an approved Compensation Philosophy Framework (as defined in Section~\ref{def:compensation-philosophy}).

The framework must:
\begin{itemize}
\item use Independently Verifiable Compensation Data (as defined in Section~\ref{def:independently-verifiable-compensation-data})
\item establish role-based target ranges rather than a single universal percentile
\item document the rationale for range placement for each role
\item preserve internal equity across roles of comparable responsibility
\item remain consistent with long-term organizational solvency
\end{itemize}

Compensation for any role must not fall below the Dignity Floor (as defined in Section~\ref{def:dignity-floor}).

Compensation may exceed the Upper End of Prevailing Market Norms only with documented justification.  
Compensation below the Upper End of Prevailing Market Norms requires documented, non-extractive rationale and must not rely on mission alignment, prestige, or implied sacrifice.

Any deviation from the applicable target range requires written justification and Governing Body review.

Mutability: Unchangeable


\subsubsection{Internal Equity and Non-Retaliation}

Compensation practices must preserve internal equity across roles of comparable responsibility and contribution.

Compensation must not be adjusted, withheld, or structured as:
\begin{itemize}
\item punishment
\item coercion
\item retaliation
\item reward for silence or compliance
\end{itemize}

Any adverse compensation action linked to reporting, dissent, or cooperation with review processes constitutes presumptive Abuse under Section~\ref{subsec:abuse}.


Mutability: Unchangeable

\subsubsection{Founder Compensation Safeguard}

The Governing Body must independently determine and document a fair market-aligned compensation recommendation for the Founder (as defined in Section~\ref{def:founder})
 using the same standards applied to all other roles.

The Founder may voluntarily petition the Governing Body to accept compensation below the recommended level.

Such a petition:
\begin{itemize}
\item must be initiated solely by the Founder
\item must be documented in writing
\item must not result in increased authority, influence, or informal consideration
\item may be withdrawn by the Founder at any time
\end{itemize}

The Governing Body must not pressure, incentivize, or expect the Founder to accept below-market compensation.

Mutability: Unchangeable

\subsubsection{Prohibition on Compensation-Based Extraction}

Compensation structures must not be used to:
\begin{itemize}
\item extract uncompensated labor
\item mask underpayment through housing, perks, or informal benefits
\item shift financial burden onto employees for stewardship reasons
\end{itemize}

Residence protections under Section~\ref{sec:residential-continuity} do not constitute compensation and must not be used to offset, justify, or reduce fair pay.


Mutability: Unchangeable


\section{Enforcement Authority}
\label{sec:enforcement-authority}

Where enforcement authority is specified, that authority governs. Where unspecified, enforcement defaults to the Governing Body under the operational manuals and is subject to the Classification, Arbitration, and Accountability provisions (Section~\ref{sec:classification-accountability}).
No enforcement action may occur without a Recorded Decision Pathway (as defined in Section~\ref{def:recorded-decision-pathway}).


For emergency actions, a Recorded Decision Pathway exists upon creation of a compliant emergency log stub as defined in Section~\ref{def:recorded-decision-pathway}, even if full documentation is completed later.


Mutability: Revisable

\section{Emergency Scope Registry}
\label{sec:emergency-scope-registry}

Emergency authority exists only as explicitly defined in this section and may be exercised solely within the categories enumerated below.

For avoidance of doubt, this section defines the exclusive categories and limits for emergency authority under this document.

The Emergency Scope Registry:
\begin{itemize}
\item defines the exclusive categories under which emergency authority may be invoked,
\item specifies the permitted actions, forbidden actions, trigger standards, stop conditions, and expiration limits for each category,
\item limits emergency authority to prevention of imminent harm only,
\item does not confer governance authority, disciplinary authority, policy authority, or contracting authority.
\end{itemize}

No emergency action may be taken absent valid authority under a Registry category.
Actions outside the Registry are unauthorized and prohibited under this document.

Each emergency invocation:
\begin{itemize}
\item must reference a valid Registry category code,
\item must reference affected Registry objects using the controlled vocabulary defined in the Registry,
\item must comply with the expiration and stop conditions of the applicable category,
\item must be logged as a Material Action using an emergency log stub as defined in Section~\ref{def:recorded-decision-pathway}.
\end{itemize}

Emergency authority must not be used to:
\begin{itemize}
\item revoke access rights, evict, ban, or trespass any person,
\item discipline staff or participants,
\item modify policy, procedure, or governance rules,
\item authorize contracts, partner commitments, or new financial obligations,
\item perform any action irreversible without Governing Body approval.
\end{itemize}

Emergency authority does not create precedent and must not be cited as authorization for future actions.

Registry entries are immutable during any emergency.
Registry entries may be added, removed, or modified only by prior Governing Body approval and must not be altered retroactively.

Operational Manuals may reference the Registry but must not expand, interpret, or narrow its scope.


\subsection{Emergency Authority Sunset}
\label{subsec:emergency-authority-sunset}   

All Emergency Authority definitions, including Emergency Scope Registry entries and any governance-approved emergency procedures, automatically expire twenty-four (24) months after approval unless explicitly reaffirmed by the Governing Body.

Reaffirmation requires:
\begin{itemize}
\item a recorded governance vote
\item confirmation that the authority remains necessary and narrowly scoped
\item revalidation or re-entry in the Emergency Scope Registry
\end{itemize}

If emergency authority expires due to non-reaffirmation, all actions must proceed only under ordinary role-scoped authority and applicable law. Any action taken to prevent imminent harm during such lapse must be logged as a Material Action and must be placed on the agenda of the next Governing Body meeting for review and reaffirmation decision.


Mutability: Unchangeable

\subsection{Continuity Authority During Quorum Loss}
\label{subsec:continuity-authority}

If the Governing Body lacks quorum and such loss persists beyond seven (7) consecutive days, a temporary Continuity Authority is automatically activated for the sole purpose of preventing imminent harm, maintaining legal compliance, and preserving animal welfare.

Continuity Authority:
\begin{itemize}
\item does not constitute governance authority
\item does not permit Material Actions except as strictly necessary to prevent imminent harm or legal violation
\item does not permit strategic decisions, compensation changes, asset disposition, land use changes, or policy modification
\item exists solely until quorum is restored
\end{itemize}

Actions under Continuity Authority are limited to:
\begin{itemize}
\item actions required to comply with law, court order, or regulatory mandate
\item actions required to maintain basic animal welfare, safety, utilities, insurance, and payroll continuity
\item actions required to prevent irreversible harm caused solely by inaction
\end{itemize}

All actions taken under Continuity Authority:
\begin{itemize}
\item must be logged as Material Actions
\item must include an explicit quorum-loss justification
\item must be reversible where practicable
\item must be reviewed by the Governing Body within thirty (30) days of quorum restoration
\end{itemize}

Continuity Authority automatically terminates upon restoration of quorum.

Any action exceeding the scope of this section constitutes prima facie evidence of Violation. 

Continuity Authority must not be used to simulate, extend, substitute for, or delay invocation of the Constitutional Repair Protocol or any Interim Mitigation authorized thereunder. Actions taken under Continuity Authority must remain strictly limited to immediate harm prevention and must not be aggregated or sequenced to achieve effects that would otherwise require Repair or Mitigation authorization.


Mutability: Unchangeable


\subsection{Registry Object Vocabulary}

Emergency log stubs must reference affected objects using only the following controlled classes:

\begin{itemize}
\item \texttt{PERSON.GUEST}, \texttt{PERSON.STAFF}, \texttt{PERSON.VOLUNTEER}, \texttt{PERSON.PARTICIPANT}
\item \texttt{ANIMAL.DOG}, \texttt{ANIMAL.OTHER}
\item \texttt{SPACE.BUILDING}, \texttt{SPACE.ROOM}, \texttt{SPACE.CORRIDOR}, \texttt{SPACE.TRAIL}, \texttt{SPACE.FIELD}, \texttt{SPACE.WATER\_FEATURE}, \texttt{SPACE.PARKING}
\item \texttt{ASSET.UTILITY.GAS}, \texttt{ASSET.UTILITY.ELECTRIC}, \texttt{ASSET.UTILITY.WATER}, \texttt{ASSET.UTILITY.HVAC}
\item \texttt{ASSET.SAFETY.FIRE\_EXT}, \texttt{ASSET.SAFETY.AED}, \texttt{ASSET.SAFETY.FIRST\_AID}, \texttt{ASSET.SAFETY.BARRICADE\_KIT}, \texttt{ASSET.SAFETY.SIGNAGE}
\item \texttt{SYSTEM.ACCESS\_CONTROL}
\item \texttt{SYSTEM.LOGGING\_PRIMARY}, \texttt{SYSTEM.LOGGING\_OFFLINE}
\end{itemize}

Free-text object references are invalid.

\subsection{E-001 - Immediate Medical Emergency (Human)}

\textbf{Trigger Standard:}  
Observable signs of life-threatening or urgent medical distress requiring immediate intervention.

\textbf{Permitted Actions:}
\begin{itemize}
\item Call emergency medical services.
\item Deploy first-aid or AED equipment within training scope.
\item Temporarily clear and control nearby space to preserve safety and dignity.
\end{itemize}

\textbf{Forbidden Actions:}
\begin{itemize}
\item Administration of medication beyond training.
\item Restraint except to prevent immediate self-harm.
\item Recording, photographing, or publicizing the event.
\end{itemize}

\textbf{Stop Condition:}  
Handoff to medical authority or stabilization with refusal of further care.

\textbf{Maximum Expiration:}  
Four (4) hours.

\subsection{E-002 - Immediate Medical Emergency (Animal)}

\textbf{Trigger Standard:}  
Animal in acute distress creating imminent risk to life or severe injury.

\textbf{Permitted Actions:}
\begin{itemize}
\item Remove immediate hazard.
\item Relocate animal to nearest safe containment or treatment-ready area.
\item Contact pre-approved emergency veterinary resources.
\end{itemize}

\textbf{Forbidden Actions:}
\begin{itemize}
\item Experimental treatment.
\item Sedation or euthanasia.
\item Irreversible medical decisions.
\end{itemize}

\textbf{Stop Condition:}  
Animal stabilized or transferred to qualified care.

\textbf{Maximum Expiration:}  
Four (4) hours.

\subsection{E-010 - Fire, Smoke, or Explosion Risk}

\textbf{Trigger Standard:}  
Visible smoke or flame, alarm activation with corroboration, or credible report of fire risk.

\textbf{Permitted Actions:}
\begin{itemize}
\item Activate alarms and contact fire services.
\item Evacuate affected zones.
\item Use fire extinguishers only if trained and safe.
\end{itemize}

\textbf{Forbidden Actions:}
\begin{itemize}
\item Re-entry without clearance.
\item Disabling alarms.
\item Improvised repairs.
\end{itemize}

\textbf{Stop Condition:}  
Fire authority clearance.

\textbf{Maximum Expiration:}  
Four (4) hours.

\subsection{E-011 - Gas Leak or Suspected Gas}

\textbf{Trigger Standard:}  
Odor of gas, detector alarm, or credible report.

\textbf{Permitted Actions:}
\begin{itemize}
\item Evacuate affected zones.
\item Restrict access using barricades or signage.
\item Contact utility emergency services and fire department.
\end{itemize}

\textbf{Forbidden Actions:}
\begin{itemize}
\item Ignition sources.
\item Repairs or investigation beyond trained checklists.
\end{itemize}

\textbf{Stop Condition:}  
Utility or fire authority clearance.

\textbf{Maximum Expiration:}  
Four (4) hours.

\subsection{E-020 - Structural or Environmental Hazard}

\textbf{Trigger Standard:}  
Observable physical condition creating imminent injury risk.

\textbf{Permitted Actions:}
\begin{itemize}
\item Temporarily close the smallest necessary area.
\item Post hazard signage and reroute traffic.
\end{itemize}

\textbf{Forbidden Actions:}
\begin{itemize}
\item Permanent closures.
\item Construction or modification work.
\end{itemize}

\textbf{Stop Condition:}  
Hazard neutralized or area rendered safe.

\textbf{Maximum Expiration:}  
Four (4) hours.

\subsection{E-030 - Violence Threat or Active Altercation}

\textbf{Trigger Standard:}  
Credible threat of violence, active physical altercation, or weapon sighting.

\textbf{Permitted Actions:}
\begin{itemize}
\item Contact emergency services.
\item Direct people to safe zones.
\item Temporarily restrict space to create separation.
\end{itemize}

\textbf{Forbidden Actions:}
\begin{itemize}
\item Eviction, banning, or trespass decisions.
\item Disciplinary actions.
\end{itemize}

\textbf{Stop Condition:}  
Law enforcement or security authority assumes control.

\textbf{Maximum Expiration:}  
Four (4) hours.

\subsection{E-040 - Missing Child or Missing Dog}

\textbf{Trigger Standard:}  
Report of missing child or dog with reasonable safety concern.

\textbf{Permitted Actions:}
\begin{itemize}
\item Initiate approved missing-person or missing-dog alert workflow.
\item Temporarily control specific exit points without revocation of rights.
\end{itemize}

\textbf{Forbidden Actions:}
\begin{itemize}
\item Tracking of adults without opt-in.
\item Public disclosure beyond approved fields.
\end{itemize}

\textbf{Stop Condition:}  
Reunion or escalation to external authorities.

\textbf{Maximum Expiration:}  
Four (4) hours.

\subsection{E-050 - Utility Failure Threatening Welfare or Safety}

\textbf{Trigger Standard:}  
Utility outage creating imminent risk to welfare or safety.

\textbf{Permitted Actions:}
\begin{itemize}
\item Relocate people or animals to safe zones.
\item Activate pre-installed backup systems where trained.
\item Contact emergency utility services.
\end{itemize}

\textbf{Forbidden Actions:}
\begin{itemize}
\item Authorizing paid emergency work beyond pre-approved agreements.
\item Infrastructure modification.
\end{itemize}

\textbf{Stop Condition:}  
Utility restoration or stabilized safe alternative.

\textbf{Maximum Expiration:}  
Four (4) hours.

\subsection{E-090 - Logging System Outage}

\textbf{Trigger Standard:}  
Primary logging system unavailable at time of action.

\textbf{Permitted Actions:}
\begin{itemize}
\item Creation of offline, tamper-evident emergency log stubs.
\item Secure storage of stubs pending transcription.
\end{itemize}

\textbf{Forbidden Actions:}
\begin{itemize}
\item Free-text-only stubs.
\item Retroactive modification or deletion.
\end{itemize}

\textbf{Stop Condition:}  
Primary logging restored.

\textbf{Maximum Expiration:}  
Seventy-two (72) hours to complete transcription.

This Mutability designation applies to the Emergency Scope Registry in its entirety, including all Registry entries and controlled vocabularies.

Mutability: Unchangeable

\section{Operations and Manuals}

Operational manuals:
\begin{itemize}
\item Must be written in IF / THEN / ELSE logic
\item Are subordinate to this document
\item May not create new powers or override prohibitions
\end{itemize}
Operational Manuals may define narrowly scoped emergency action procedures solely to prevent imminent harm, provided such procedures create no new authority, confer no discretion beyond necessity, and remain fully subject to logging, review, and classification requirements of this document.


Mutability: Adaptive

\section{Assets, Land, and Dissolution}

\subsection{Asset Stewardship}
Assets are held exclusively for charitable purposes. No asset may be distributed for private gain.

\subsection{Land Use}
Land use does not imply ownership. No land arrangement may enable extraction or speculative benefit.

\subsection{Dissolution}
Assets must transfer to a qualified charitable entity with aligned purposes. 

Mutability: Unchangeable

\section{Transparency and Overrides}

\subsection{Material Actions}

All Material Actions (as defined in Section~\ref{def:material-action}) must be:

\begin{itemize}
\item Logged
\item Attributable
\item Reviewable
\end{itemize}

Overrides require documented justification and retrospective review. 

Mutability: Revisable

\subsection{Materiality Challenge Protocol}
\label{def:materiality-challenge}

If reasonable disagreement exists as to whether an action is a Material Action, any voting member or affected party may file a Materiality Challenge in writing.

Upon a Materiality Challenge:
\begin{itemize}
\item the action must be treated as a Material Action unless and until resolved,
\item the Governing Body (or a designated subcommittee if authorized by Operational Manuals) must issue a written determination within seventy-two (72) hours,
\item the determination must be logged and must cite the specific criteria in Section~\ref{def:materially-affects} and Section~\ref{def:material-action}.
\end{itemize}

Failure to issue a timely determination constitutes a governance process failure and automatically triggers Classification Panel review for procedural remediation only.

Mutability: Unchangeable


\subsection{Temporary Collective Hardship Covenant}
\label{subsec:hardship-covenant}

This Covenant exists solely to permit temporary organizational survival during severe, documented financial distress without transferring that distress onto specific individuals through coercion or exploitation.

Invocation of this Covenant is permitted only if all of the following conditions are met:

\begin{itemize}
\item the Governing Body determines, based on documented financial evidence, that continued operation under standard compensation obligations would result in imminent insolvency or forced closure
\item all reasonable non-extractive alternatives have been evaluated and documented
\item the Covenant is approved by a Supermajority vote as defined in Section~\ref{def:supermajority}

\item the scope, duration, and financial rationale are fully disclosed to all affected persons
\end{itemize}

When invoked, the Covenant operates under the following constraints:

\begin{itemize}
\item participation is voluntary for each employed person
\item no person may be compelled, pressured, or incentivized to accept reduced compensation
\item refusal to participate must not result in retaliation, termination, role degradation, or adverse treatment
\item any reduction applies symmetrically across participating roles, including leadership and the Founder
\item no compensation may fall below the Dignity Floor (as defined in Section~\ref{def:dignity-floor})

\item the Covenant automatically expires after a fixed duration not exceeding ninety days unless explicitly reaffirmed
\end{itemize}

The Covenant must not be used to:
\begin{itemize}
\item permanently alter compensation norms
\item reset market benchmarks downward
\item extract unpaid labor
\item replace fair compensation obligations
\end{itemize}

Any misuse, misrepresentation, coercion, or selective application of this Covenant constitutes prima facie evidence of Abuse under Section~\ref{subsec:abuse}.


Mutability: Unchangeable


\section{Protected Residential Continuity}
\label{sec:residential-continuity}

\subsection{Founder Continuity Residence}

The organization may provide a non-transferable on-site residence to the Founder (as defined in Section~\ref{def:founder}) solely to preserve organizational continuity, security, and long-horizon stewardship.

This residence:
\begin{itemize}
\item confers no ownership interest
\item confers no equity, inheritance, or succession right
\item is not transferable
\item must qualify as an Independently Valued Housing Benefit (as defined in Section~\ref{def:independently-valued-housing})
\end{itemize}

Provision of this residence requires:
\begin{itemize}
\item an independent fair-market rental valuation
\item a documented business-necessity rationale adopted by the Governing Body
\item conflict-of-interest review with the Founder fully recused
\item internal disclosure to staff
\item donor-facing disclosure at a level sufficient to avoid misrepresentation
\end{itemize}

\subsection{Periodic Review}

The Founder Continuity Residence must be reviewed no less frequently than once every five (5) years to confirm that:
\begin{itemize}
\item the business-necessity rationale remains valid
\item the valuation remains current
\item no private benefit has accrued beyond what is permitted
\end{itemize}

Failure to complete or document a required review automatically suspends the protection until cured.

\subsection{Survivor Occupancy Transition}

Upon the Founder’s death, a surviving spouse may continue occupying the residence for life, subject to the constraints below, and must not be displaced solely due to the Founder’s death.

Such continued occupancy:
\begin{itemize}
\item is a residency protection only and conveys no ownership, governance authority, income, or transferability,
\item must not materially interfere with operations or safety,
\item must not violate binding prohibitions under this document or applicable law,
\item must preserve dignity and housing stability for the surviving spouse.
\end{itemize}

The Governing Body may impose reasonable, documented conditions solely to preserve safety, operational continuity, or legal compliance, but must not impose time limits, coercive terms, or displacement absent such necessity.

\subsection{Service-Based Occupancy}

A person raised in the Founder’s household may occupy the residence only while holding a full-time, bona fide working position with the organization under standard employment terms.

This occupancy:
\begin{itemize}
\item terminates automatically upon cessation of full-time service
\item confers no ownership or succession right
\item is subject to the same performance and conduct standards as any other employee
\end{itemize}

\subsection{Non-Expansion and Anti-Precedent Rule}

No residential right under this section:
\begin{itemize}
\item may be extended to additional persons
\item may be generalized beyond its documented necessity
\item may be used as precedent for future housing arrangements
\end{itemize}

\subsection{Abuse Disqualification}

Any final finding of Abuse under Section~\ref{subsec:abuse} immediately terminates all protections under this section for the offending individual.

Mutability: Unchangeable


\section{Amendment and Mutability Rules}

Each section declares its mutability class.

Where nested sections or subsections declare different mutability classes, the most specific mutability designation governs.
\begin{itemize}
\item Unchangeable sections may not be amended
\item Revisable sections require governance approval
\item Adaptive sections may evolve within constraints
\end{itemize}

\subsection{Foundational Phase Operability}
\label{subsec:foundational-phase-operability}

New Doggerland may operate under a Foundational Phase to permit safe and accountable operation while specified governance mechanisms, operational manuals, logs, or administrative systems are not yet fully instantiated.

A Foundational Phase may be activated only by a Recorded Decision Pathway (as defined in Section~\ref{def:recorded-decision-pathway}) approved by Supermajority vote (as defined in Section~\ref{def:supermajority}).

A Foundational Phase activation record must include all of the following:
\begin{itemize}
\item the specific governance mechanisms, procedures, logs, systems, or documents that are not yet instantiated,
\item the factual reason each item is not yet instantiated,
\item the interim handling method for each missing item, stated mechanically and without discretion,
\item the specific instantiation plan and target dates for each missing item,
\item the start date and automatic expiration date for the Foundational Phase.
\end{itemize}

A Foundational Phase must not:
\begin{itemize}
\item waive, suspend, or narrow any Unchangeable provision,
\item expand authority beyond what is explicitly granted in this document,
\item permit actions that would otherwise require emergency authority or repair authority,
\item bypass required recusals, conflict disclosures, or vote thresholds that are practicable to execute.
\end{itemize}

Interim handling methods adopted under this subsection:
\begin{itemize}
\item are non-precedential,
\item must be limited to the narrowest scope necessary to maintain operability,
\item must be logged as Material Actions if they meet the Material Action definition in Section~\ref{def:material-action},
\item automatically terminate upon instantiation of the corresponding missing item, even if the Foundational Phase remains active for other items.
\end{itemize}

A Foundational Phase automatically expires no later than twenty-four (24) months after activation.

A Foundational Phase may be renewed only by an Extraordinary Process (as defined in Section~\ref{def:extraordinary-process}) and only if the renewal record:
\begin{itemize}
\item identifies each remaining missing item,
\item explains why instantiation has not occurred,
\item narrows interim handling scope where possible, and
\item sets a new fixed expiration date not exceeding twelve (12) months.
\end{itemize}

Mutability: Revisable only by Extraordinary Process


Mutability: Unchangeable

\section{Interpretation}

This document must be interpreted narrowly. Where ambiguity exists, interpretation must favor:
\begin{itemize}
\item Mission preservation
\item Non-exploitation
\item Transparency
\end{itemize}

No interpretation may expand authority beyond what is explicitly granted. 

Mutability: Unchangeable

\subsection{Interpretation Tie-Break Protocol}

If two or more interpretations of a provision remain plausible after applying plain meaning and contextual reading, the following tie-break hierarchy must be applied in order:

\begin{itemize}
\item the interpretation that best preserves mission integrity
\item the interpretation that minimizes potential harm to welfare and dignity
\item the interpretation that imposes the narrowest grant of authority
\item the interpretation that preserves the greatest number of Unchangeable constraints
\end{itemize}

No interpretation may be adopted solely on the basis of convenience, efficiency, precedent, or institutional preference.

If a tie remains after applying this hierarchy, the matter must be escalated to the Classification Panel for documented resolution.

Mutability: Unchangeable


\subsection{Early-Stage Governance Principle}

The governance framework of New Doggerland is intentionally designed to be fully operational prior to organizational scale or reputational maturity. This sequencing reflects the principle that safety, accountability, and control must precede growth rather than follow it.

Future decisions regarding expansion, funding structure, or operational tempo shall be evaluated against this principle, with preference given to preserving governance integrity over accelerating outcomes.


\section{Failure Handling}

Mistakes are expected. Bad-faith actions, concealment, or retaliation are prohibited. Errors addressed transparently must not be punished solely for occurrence. 

Mutability: Revisable

\section{Stewardship and Permanence Constraints}

\begin{itemize}
\item Long-Horizon Duty: No short-term gains over long-term stewardship
\item Non-Extractive Principle: No structure may extract value from land, animals, or people
\item Maintenance Obligation: No adoption of assets that can’t be reasonably maintained
\item Durability Preference: Favor repairable and durable options
\item Stewardship Over Optimization: Cost or growth cannot override core duties
\item Adaptive Compliance: Methods may change, but harms may not be allowed
\end{itemize}

Mutability: Unchangeable

\section{Classification, Arbitration, and Accountability}
\label{sec:classification-accountability}


\subsection{Action Classification}
All material actions must be classified as Mistake, Violation, or Abuse.  

Mutability: Unchangeable

\subsection{Mistake}
Good faith + reasonable belief + prompt disclosure + non-recurrence = Mistake.  

Requires correction and documentation only.  

Mutability: Unchangeable

\subsection{Violation}
Intentional bypass, concealment, reckless disregard, or persistence after correction = Violation.  

Requires formal review.  

Mutability: Unchangeable

\subsection{Abuse}
\label{subsec:abuse}

Meets Violation criteria + exploit intent, retaliation, repetition, or credible harm.  

Requires escalation and may trigger authority removal.  

Mutability: Unchangeable

\subsection{Classification Authority}
A three-member panel is required. No person may classify their own conduct.  

Mutability: Unchangeable

\subsection{Panel Selection}
Each panel member is selected independently by:
\begin{itemize}
\item Governance
\item Stewardship
\item Staff / External
\end{itemize}

Mutability: Revisable

\subsection{Relationship to Values}
If values materially relate to harm, the panel must assess whether they were ignored.  

Mutability: Unchangeable

\subsection{Classification Standards}
Only logs, facts, and credible testimony may be considered.  

Belief or values alone are not sufficient.  

Mutability: Unchangeable

\subsection{Burden of Proof}
\begin{itemize}
\item Mistake = Preponderance
\item Violation = Clear evidence
\item Abuse = Clear and convincing
\end{itemize}

If the threshold is not met, classification defaults to the lowest applicable level.  

Mutability: Unchangeable

\subsection{Testimony Definition}
Testimony must be first-hand, internally consistent, and corroborated.  

Uncorroborated testimony alone cannot support an Abuse finding.  

Mutability: Unchangeable

\subsection{Appeals}
Each classified party may appeal once to a pre-approved External Governance Arbiter (as defined in Section~\ref{def:external-governance-arbiter}).


Mutability: Revisable

\subsection{Panel Term Enforcement}
Panelists serve staggered 24-month terms, max two consecutive terms.  
Expired panelists are automatically disqualified.  

Mutability: Unchangeable

\subsection{Quorum Safeguards}
Quorum loss pauses classification and prohibits irreversible decisions.  

Mutability: Unchangeable

\subsection{Arbiter Pool Integrity}
\begin{itemize}
\item Must include at least 1/3 external origin
\item No single source can control majority
\item Published with source and conflict info
\item Governance may remove for cause, but not unilaterally replace
\item Replacements must originate from the same category
\end{itemize}

Mutability: Unchangeable

\subsection{Arbiter Assignment Randomization}

Where this document requires an External Governance Arbiter, the Arbiter must be assigned by a documented random selection process from the eligible Arbiter pool, unless random selection is infeasible due to conflicts, unavailability, or time-critical constraints.

If random selection is infeasible, the appointing body must document:
\begin{itemize}
\item the specific infeasibility reason,
\item the conflicts screened,
\item the selection method used, and
\item why the selected Arbiter best preserves independence under Section~\ref{def:external-governance-arbiter}.
\end{itemize}

Mutability: Unchangeable


\subsection{Stewardship Backstop}
If the Governing Body fails to rotate panelists, the Steward’s Office (as defined in Section~\ref{def:stewards-office}) appoints a temporary interim member strictly for continuity purposes and not eligible for a full term.


Mutability: Unchangeable

\subsection{Emergency Good-Faith Presumption}
When emergency authority is invoked to prevent imminent harm, subsequent review must begin with a presumption of good faith unless evidence indicates concealment, reckless disregard, or exploit intent.

This presumption does not prevent review, classification, or consequence, but it prohibits adverse classification based solely on frequency, urgency, or outcome absent supporting evidence.

Repeated emergency invocations by the same actor within a rolling review period automatically trigger a mandatory comparative analysis and panel review. Frequency alone must not determine classification, but it must require the panel to assess patterns, decision-pathway quality, and scope compliance.


Mutability: Unchangeable

\subsection{Pattern-Based Corroboration}

Repeated, independently logged actions that align in time, method, or target may be treated as corroborative evidence even if no single incident independently meets the Abuse threshold.

Patterns may corroborate intent, recklessness, or concealment when:
\begin{itemize}
\item actions recur under substantially similar circumstances
\item affected parties or targets overlap
\item logs, timing, or decision pathways show consistent deviation
\end{itemize}

Pattern-based corroboration must not rely on subjective impressions alone and must be grounded in recorded facts.

Mutability: Unchangeable

\subsection{Anti-Tamper Presumption}

Any intentional alteration, suppression, destruction, or fabrication of logs, records, or audit trails constitutes prima facie evidence of Violation.

If tampering materially impairs classification, the presumption escalates to Abuse unless the actor demonstrates a credible, non-exploitative cause.

Good-faith errors that are promptly disclosed and corrected do not constitute tampering.

Emergency log stubs must be written to an append-only, tamper-evident record system meeting policy-defined integrity requirements. Deletion or modification must be detectably evident upon review.


Mutability: Unchangeable

\subsection{Residence Protection Disqualification Upon Abuse}

Any finding of Abuse under this Section automatically terminates all residence protections granted under Section~\ref{sec:residential-continuity} for the offending individual.


Termination under this subsection:
\begin{itemize}
\item is mandatory and non-discretionary
\item applies regardless of tenure, role, or relationship
\item does not require a separate vote or process
\item takes effect upon final classification or exhaustion of appeal
\end{itemize}

This disqualification does not affect the rights of non-offending parties unless independently implicated by classification.

Mutability: Unchangeable

\subsection{Whistleblower Protection and Anti-Retaliation Presumption}

Any adverse action taken against a person for reporting, cooperating with, or reasonably attempting to report a concern in good faith constitutes presumptive Abuse.

Adverse actions include, but are not limited to:
\begin{itemize}
\item termination, demotion, or discipline
\item intimidation, threats, or coercion
\item housing displacement or access restriction
\item isolation, role stripping, or reputational harm
\end{itemize}

The presumption of Abuse applies unless the actor demonstrates by clear evidence that the action was:
\begin{itemize}
\item unrelated to the report or cooperation
\item proportionate
\item independently justified
\end{itemize}

Good-faith but mistaken reports do not constitute misconduct and must not trigger adverse action.

Mutability: Unchangeable


\section{Constitutional Repair Protocol}

\subsection{Purpose}
This protocol exists solely to prevent unavoidable harm caused by continued enforcement of an Unchangeable provision under materially changed conditions.

This protocol must not be used to optimize convenience, efficiency, growth, or authority.

Mutability: Unchangeable

\subsection{Trigger Threshold}
The Constitutional Repair Protocol may be invoked only if all of the following conditions are met:
\begin{itemize}
\item continued enforcement of a specific Unchangeable provision would cause substantial and demonstrable harm
\item the harm cannot be mitigated through interpretation, operational adjustment, or use of emergency authority
\item the harm arises from material changes in factual or legal conditions not reasonably foreseeable at the time of adoption
\end{itemize}

Mutability: Unchangeable

\subsection{Invocation Requirements}
Invocation requires all of the following:
\begin{itemize}
\item a supermajority vote of not less than eighty percent of the full Governing Body
\item written findings specifying the exact provision at issue, the harm caused, and why no lesser remedy is sufficient
\item unanimous approval by a pre-approved External Governance Arbiter panel composed solely of eligible Arbiters as defined in Section~\ref{def:external-governance-arbiter}

\end{itemize}

Mutability: Unchangeable

\subsection{Scope Limitation}
Any modification under this protocol must:
\begin{itemize}
\item be limited strictly to the minimum change necessary to remove the identified harm
\item preserve all other Unchangeable provisions without alteration
\item create no new powers, rights, or entitlements beyond what is required to eliminate the harm
\end{itemize}

Mutability: Unchangeable

\subsection{Interim Harm Mitigation Pending Repair}

If the Trigger Threshold in this Protocol is met but the Delay and Disclosure period has not yet elapsed, the Governing Body may authorize a strictly temporary Interim Mitigation solely to prevent the specific harm identified in the written findings.

Interim Mitigation:
\begin{itemize}
\item does not amend, suspend, or waive any Unchangeable provision
\item is permitted only to reduce the identified harm pending the effective date of an approved Repair
\item must be the minimum operational deviation necessary to prevent harm
\item must be time-limited to not more than thirty (30) days per authorization
\item must be renewed only by the same vote threshold required for Invocation Requirements under this Protocol
\item must be reversible where practicable
\item must be logged as a Material Action with an explicit INTERIM-MITIGATION reason code and citation to the specific Repair findings
\end{itemize}

Interim Mitigation must not:
\begin{itemize}
\item create new powers, entitlements, compensation rights, housing rights, or governance authority
\item authorize asset disposition, land use changes, policy modification, or any action prohibited under Continuity Authority
\end{itemize}

All Interim Mitigation authorizations and renewals must be included in the public disclosure materials for the Repair.

Mutability: Unchangeable


\subsection{Delay and Disclosure}
Approved modifications must be subject to a mandatory public disclosure and delay period of not less than sixty days prior to effect.

The disclosure must include the full text of the proposed change, the recorded votes, and the written findings.

Mutability: Unchangeable

\subsection{Post-Repair Review}
Within twelve months of any repair taking effect, the Governing Body must commission an independent review assessing:
\begin{itemize}
\item whether the repair successfully eliminated the identified harm
\item whether unintended consequences occurred
\item whether the repair can be narrowed or reversed
\end{itemize}

The review must be published in full.

Mutability: Unchangeable


\end{document}
