
\documentclass[11pt]{article}
\usepackage{ndstyle}
\setcounter{secnumdepth}{2}
\setcounter{tocdepth}{2}

\title{New Doggerland \\ Governance Schedules \\ Schedule C - Delegation and Approval Matrix}
\author{}
\date{}

\begin{document}
\maketitle
\tableofcontents
\newpage

\noindent
\textbf{Status:} Binding Governance Schedule \\
\textbf{Authority:} Adopted by the Governing Body pursuant to the Consolidated Governance Document \\
\textbf{Scope:} Defines who may approve which actions, up to what limits, and under what constraints \\
\textbf{Precedence:} Subordinate to the Consolidated Governance Document; controlling where referenced \\
\textbf{Mutability:} Revisable only by Extraordinary Process

\section{Purpose}

This Schedule defines the exclusive delegation and approval limits for financial, contractual, operational, and emergency-related actions within New Doggerland.

No person or role may approve, authorize, or commit the organization beyond the limits explicitly stated in this Schedule.

Silence in this Schedule does not confer authority.

\section{General Principles}

All delegation under this Schedule is subject to the following constraints:

\begin{itemize}
\item Delegation defines \emph{who may approve}, not \emph{what is permissible}.
\item All actions remain subject to Mission Lock, welfare, dignity, and safety constraints.
\item Any action exceeding these limits constitutes a Material Action.
\item Emergency authority does not expand approval limits beyond this Schedule.
\end{itemize}

\section{Approval Roles}

For purposes of this Schedule, approval authority is limited to the following roles:

\begin{itemize}
\item Governing Body (collective)
\item Governing Body Chair (procedural only; no approval authority unless otherwise specified)
\item Executive Director (or equivalent senior executive)
\item Designated Finance Officer
\item Role-Scoped Staff Lead (as documented in role charter)
\end{itemize}

No other role may exercise approval authority.

\section{Financial Expenditure Approvals}

\subsection{Routine, Budgeted Expenditures}

\begin{center}
\begin{tabular}{|l|l|l|}
\hline
\textbf{Amount (USD)} & \textbf{Approver} & \textbf{Conditions} \\
\hline
Up to \$1{,}000 & Role-Scoped Staff Lead & Budgeted; logged \\
\$1{,}001 -- \$5{,}000 & Executive Director & Budgeted; logged \\
\$5{,}001 -- \$25{,}000 & Finance Officer + Executive Director & Budgeted; logged \\
Over \$25{,}000 & Governing Body & Supermajority; Recorded Decision Pathway \\
\hline
\end{tabular}
\end{center}

\subsection{Unbudgeted or Cumulative Expenditures}

Any unbudgeted expenditure, or cumulative related expenditures exceeding \$10{,}000 within a rolling 90-day period, requires Governing Body approval and constitutes a Material Action.

\section{Contractual Commitments}

\begin{center}
\begin{tabular}{|l|l|l|}
\hline
\textbf{Contract Term / Value} & \textbf{Approver} & \textbf{Conditions} \\
\hline
Term $\leq$ 6 months and \$10{,}000 & Executive Director & Due diligence complete \\
Term $>$ 6 months or \$10{,}001 -- \$50{,}000 & Governing Body & Majority vote; logged \\
Over \$50{,}000 or $>$ 12 months & Governing Body & Supermajority; RDP required \\
\hline
\end{tabular}
\end{center}

No contract may be executed without documented conflict-of-interest screening.

\section{Personnel and Compensation Actions}

The following actions always constitute Material Actions and require Governing Body approval:

\begin{itemize}
\item Creation of new paid roles
\item Compensation adjustments exceeding 10\% annually
\item Any reduction implicating the Dignity Floor
\item Severance agreements or settlements
\end{itemize}

Routine, role-scoped hiring within approved headcount and compensation bands may be approved by the Executive Director, provided full documentation is logged.

\section{Asset Disposition and Encumbrance}

Any sale, lease, encumbrance, or disposal of assets with a fair market value exceeding \$5{,}000 requires Governing Body approval.

Any land-related transaction constitutes a Material Action regardless of value.

\section{Emergency Actions}

Emergency actions may be approved only within the scope defined in Schedule B.

Emergency authority:
\begin{itemize}
\item does not permit financial commitments exceeding \$2{,}500,
\item does not permit contract execution,
\item must be logged immediately via emergency log stub.
\end{itemize}

Any emergency action exceeding these limits constitutes a Violation.

\section{Banking and Account Access}

Authority to access, initiate, or approve banking transactions is restricted as follows:

\begin{itemize}
\item Dual-signature required for all transactions over \$5{,}000
\item No individual may both initiate and approve the same transaction
\item Governing Body members must not have unilateral account control
\end{itemize}

Account access assignments must be reviewed annually and upon role changes.

\section{Prohibited Delegations}

The following may not be delegated under any circumstances:

\begin{itemize}
\item Amendment of governance documents
\item Waiver of Unchangeable provisions
\item Approval of Related-Party Transactions
\item Override of safety, welfare, or dignity protections
\end{itemize}

\section{Interpretive Rule}

This Schedule must be interpreted narrowly. Where ambiguity exists, interpretation must favor:

\begin{itemize}
\item the lowest authority capable of acting,
\item escalation rather than expansion,
\item classification as a Material Action.
\end{itemize}

\section{Amendment and Review}

This Schedule may be amended only through an Extraordinary Process as defined in the Consolidated Governance Document.

All amendments must be adopted by recorded vote, versioned, and published alongside prior versions.

\end{document}
